%%version 1.2
\documentclass[a4paper, 14pt]{article}
\usepackage[utf8]{inputenc}
\usepackage[T2A]{fontenc}
\usepackage[english, russian]{babel} 
\usepackage{amscd, amsmath, amssymb, amsthm, enumerate, indentfirst, longtable, ifthen} %% addlibrary rom mathematic 
\usepackage{ragged2e}
\usepackage{microtype}
\usepackage{indentfirst}
\usepackage{setspace}
\setstretch{1.15} % interval 
\usepackage{fancyhdr} %colontitul
 %picture 
\RequirePackage{caption}
\DeclareCaptionLabelSeparator{defffis}{. }
\captionsetup{justification=centering,labelsep=defffis}
\usepackage{graphicx}
\graphicspath{{pict/}}
 
\DeclareGraphicsExtensions{.png,}
 %%end picture 
\usepackage{geometry}
\geometry{
 	tmargin=15mm,         % Up 
 	bmargin=15mm,         % Down 
 	lmargin=15mm,         % left поле
 	rmargin=15mm,         % right  
 }

\usepackage{graphicx,xcolor} % графика для svg
\graphicspath{{fig/}}

%first page colontitul
%				\fancypagestyle{firststyle} 	{ %новый стиль%			
%					\fancyfoot{}
%					\fancyfoot offset[R]{-12cm} %так можно регулировать ширину колонтитула
%					\fancyfoot [FR]{\thepage}
%					\fancyfoot[LO]{Copyright \textcopyright\ (Sumus Team) All rights reserved.}
%					\renewcommand{\footrulewidth}{0 mm} %% толщина отделяющей полоски сверху
%					\renewcommand{\headrulewidth}{0pt}
%					\renewcommand{\footrulewidth}{0.5pt}
%					\addtolength{\headheight}{0.5pt} % оставляем место для линейки
%				}
			


			

%% new command 
\newcommand{\jj}{\righthyphenmin=20 \justifying}
%opening
\date{27 Апреля 2018 г.}
\title{АЛГОРИТМ ДОСТИЖЕНИЯ КОНСЕНСУСА ДЛЯ БОЛЬШИХ БЛОКЧЕЙН СЕТЕЙ}
\author{a@sumus.team, k@sumus.team, rr@sumus.team}

\begin{document}

						\thispagestyle{empty}
									\pagestyle{fancy}
									% этим мы убеждаемся, что заголовки глав и 
									% разделов используют нижний регистр.
									%%				\renewcommand{\chaptermark}[1]{\markboth{#1}{}}
									%%\renewcommand{\sectionmark}[1]{\markright{\thesection\ #1}}
									\fancyhf{}  % убираем текущие установки для колонтитулов
									\fancyfoot[LE,RO]{\bfseries\thepage}
									\fancyfoot[LO]{\textcopyright\ Sumus Team. All rights reserved. 2018}
									%%\fancyfoot[RE]{\bfseries\leftmark}
									\renewcommand{\headrulewidth}{0pt}
									\renewcommand{\footrulewidth}{0.5pt}
									\addtolength{\headheight}{0.5pt} % оставляем место для линейки
									\fancypagestyle{plain}{%
										\fancyhead{} % на обычных страницах убираем колонтитулы
										%%					\fancyfoot{} % на обычных страницах убираем колонтитулы
											\fancyhead{}
											\renewcommand{\headrulewidth}{0pt}
											\renewcommand{\footrulewidth}{0.5pt}
											\addtolength{\headheight}{0.5pt} % оставляем место для линейки
								}

\maketitle

\begin{abstract} 
\jj Предлагается новый алгоритм достижения консенсуса stake distributed Byzantine Fault Tolerant (sdBFT), позволяющий увеличить на несколько порядков количество узлов сети, участвующих в достижении консенсуса, по сравнению с существующими алгоритмами семейства BFT и существенно повысить скорость транзакций.
\end{abstract}

\section{Введение}\addcontentsline{toc}{section}{Введение}
При создании монеты биткойн и технологии блокчейн Сатоши Накамото использовал в качестве алгоритма консенсуса алгоритм доказательства работы, так называемый Proof-of-Work (PoW) \cite{key-1}. С ростом популярности технологии блокчейн проявились особенности алгоритма PoW, связанные с низкой скоростью транзакций и, соответственно, их высокой стоимостью.
\\
\indent Многие эксперты из IT-индустрии создали новые либо адаптировали уже существующие алгоритмы консенсуса, устраняющие слабые стороны PoW. В результате появились алгоритмы консенсуса PoS, DPoS, LPoS, PoE, PoIT, pBFT. Данной статьей мы постарались внести свой вклад в развитие алгоритмов консенсуса.
\\
Необходимость создания собственного алгоритма консенсуса возникла при попытке создать блокчейн, отвечающий следующим требованиям к сети блокчейн:
\begin{enumerate}
	\item Время создания нового блока не более 1 мин.
	\item	Тип сети блокчейн - корпоративный.
	\item	Общее количество узлов, которые могут принять участие в выработке консенсуса, может меняться от $10^{3}$ до $10^{4}$.
	\item	Высокая скорость транзакций - не менее $10^{3}$ транзакций в секунду.
	\item	Реализации алгоритма не требует существенных энергозатрат.
	\item	Реализация алгоритма блокчейна не должна требовать существенных вычислительных, по сравнению с блокчейнами PoW, мощностей.
\end{enumerate}
	
	Ранее предложенные алгоритмы консенсуса признаны авторами не удовлетворяющими перечисленные выше требования по следующим причинам.
\\
	\indent Консенсус PoW - самый первый тип консенсуса, реализованный в блокчейне валюты Bitcoin. Консенсус отличается невысокой скоростью закрытия блока и малой скоростью транзакций, согласно приведенным данным \cite{key-2} скорость транзакций валюты Bitcoin составляет всего 7 транзакций в секунду. Формирование блока блокчейна при консенсусе PoW требует значительных вычислительных ресурсов. Причем чем больше у участника консенсуса вычислительных ресурсов, тем выше вероятность для него сформировать блок, что приводит к непроизводительной вычислительной гонке между участниками консенсуса.
\\
	\indent Консенсус Proof-of-Stake (PoS) и его вариации DPoS, LPoS были предложены, чтобы решить проблемы консенсуса PoW, связанные с высокими вычислительными издержками и малой скоростью закрытия блока транзакций \cite{key-3}. Несмотря на высокую скорость закрытия блока и низкие требования к аппаратным ресурсам, у алгоритма PoS есть недостатки. Проблемой PoS является централизация монет (ресурсов системы). Пользователь блокчейна, имеющий максимальный объем ресурсов системы, получает еще больше монет за оказание услуг по подтверждению блока. Следовательно, количество узлов, участвующих в консенсусе, будет эволюционно уменьшаться, что приведет к несоответствию требованиям к блокчейну, изложенным выше. 
\\
	\indent 	Консенсус pBFT - еще одна альтернатива алгоритму консенсуса PoW. В мире предложено несколько реализаций консенсуса pBFT, один из них - "Honey Badger" \cite{key-4}. Как показано на рис. 1 реализация pBFT консенсуса работает тем лучше, чем меньше количество узлов, участвующих в консенсусе.
\\

\begin{figure}[!h]
	\centering
%	\def\svgwidth{15cm} % изменить размер
	\input{fig/f1_v3.pdf_tex}
	\caption{Графики зависимости времени задержки формирования блока от скорости поступающих транзакций.}
	\label{fig:image}
\end{figure}

На рисунке ~\ref{fig:image} показан график зависимости времени задержки формирования блока от скорости поступающих транзакций, где Nodes/Tolerance - соотношение общего количества узлов, достигающее консенсуса, и числа узлов, не достигших консенсуса. 
	
	Кривая 1 показывает изменение времени закрытия блока в зависимости от скорости поступающих транзакций для 32 узлов, кривая 2 - для 40 узлов, кривые 3-6 - для 48, 56, 64 и 104 узлов соответственно.
	
	Наиболее эффективно консенсус работает при количестве узлов не превышающим 40, скорость транзакций для такого числа узлов достигает $2 \cdot 10^4$  транзакций в секунду, при этом время закрытия блока не превышает 40 секунд.

	Если количество узлов превышает 60, кривые 5 и 6, то скорость транзакций для такого числа узлов не превышает $0.5 \cdot 10^3$ транзакций в секунду, при этом время закрытия блока превышает 100 секунд. Время консенсуса для 104 узлов достигало 6 минут. 
	
	Авторы предлагают другой подход для решения проблемы потери производительности в алгоритме pBFT, реализованный в алгоритме stake distributed Byzantine Fault Tolerant (sdBFT) консенсуса.
	
	\section{Основные положения и допущения sdBFT}\addcontentsline{toc}{section}{sdBFT}

Мы исходим из того, что задача достижения консенсуса в первом приближении сводится к получению участниками распределенной системы согласованного решения в случае, если некоторое их количество не приняло участие в согласовании решения. Это может произойти по следующим причинам:

\begin{enumerate}
	\item 	Ошибка при передаче сообщения о принятии решения одним из участников.
	 \item 	Слишком медленная передача сообщения о принятии решения одним из участников.
	\item Сбой в работе участника в системе.
	\item Введение в заблуждение при принятии решения в системе, как умышленное, так и неумышленное.
\end{enumerate}

Если рассматривать причины 1-3 неучастия в выработке консенсуса, то для принятия решения достаточно чтобы выполнялось условие $n>m+1$ \cite{key-5}, где $m$ \textendash{} количество участников, которые не участвовали в консенсусе, а $n$ \textendash{} количество участников принявших решение. В случае появления в системе участников, не участвующих в консенсусе по четвертой причине задача сводится к задаче византийских генералов, которая имеет решение, когда
\begin{equation}
n>3\cdot m
\end{equation}

Если вместо $n$ генералов рассмотреть $n$ узлов блокчейна, участвующих в выработке консенсуса, то очевидно, что эта задача подобна задаче византийских генералов. Для решения проблемы роста времени достижения консенсуса предлагается из конечного множества узлов $B_{n}$ ,мощность этого множества $\overline{B}_{n}=n$, выделять подмножество $\overline{B}_{n^{'}}\;(B_{n^{'}}=n^{'})$ и исходя из предположения о равномерности распределения свойств узлов во всей сети блокчейн решать задачу не на $B_{n}$, а на $B_{n^{'}}$ при $n\gg n^{'}$. Общее количество узлов во всей сети блокчейн положим равным $N$. Обозначим множество этих узлов $A_{N}$.

Пусть задана функция:

\begin{equation}
d:T\mapsto Y_{B_{n}},\;d=d(t),\;t\in T
\end{equation} 

где $t$ \textendash{}  независимая переменная, соответствующая текущему времени. Будем считать, что значение $d\in Y_{B_{n}}$ функции (2) соответствует текущему состоянию $B_{n}$ в момент $t$. 

Пусть задана функция:

\begin{equation}
f:Y_{B_{n}}\mapsto\mathbb{N_{\mathit{N}}},\;f=f(d),\;d\in Y_{B_{n}}
\end{equation}

где $\mathbb{N}_{\mathit{N}}$ \textendash{}  конечное подмножество множества натуральных чисел $\mathbb{N}$, $\overline{\mathbb{N}}_n=N$.

Пусть из $B_{n}$ случайным образом выбрано подмножество $B_{n'}$, причем $n'$ задаётся. Тогда получим

\begin{equation}
B_{n'}\subset B_{n}\subseteq A_{N}
\end{equation}

Будем считать, что $Y_{B_{n{'}}}$ \textendash{}  множество значений всех $d$, соответствующих всем текущим состояниям $B_{n{'}},\;Y_{B_{n{'}}}\subseteq Y_{B_{n}}$ .

Пусть функция $f$ отображает $Y_{B_{n{'}}}$  на множество $\mathbb{N}_{n{'}},\;\overline{\mathbb{N}}_{n{'}}=n{'}$. Будем считать, что номера узлов, полученных таким отображением есть $j_{k},\;k=1,\ldots,n{'}$. Будем полагать, что $j_{\hat{k}}$ \textendash{} номер назначаемого мастер-узла, $1\leq\hat{k}\leq n{'}$. Если $b$ \textendash{} формируемый блок, в отношении которого в некоторый момент времени$t{'}$множество узлов $B_{n{'}}$ стремится достичь консенсуса, то функцию хеширования SHA-3 \cite{key-6} над этим блоком, обозначим $H(b)$, а ее значения обозначим $h$. Тогда результат вычисления электронной подписи, например, по алгоритму EdDSA\cite{key-7} с параметрами эллиптической кривой edwards25519\cite{key-7} будет равен $s=sig(h)$ .

\section{Описание алгоритма}\addcontentsline{toc}{section}{Algo}

	\begin{enumerate}
		\item Пусть в момент времени $\hat{t}\in[t,t^{'})$ узел с номером $k\;(1\leq k\leq N)$ осуществляет запись $I$ в блокчейн $B_{n}$.
		\item  Выберем все$j_{k}$, включая $j_{\hat{k}}$  с помощью функции $f$. Выработка консенсуса осуществляется на полуинтервале $[t,t^{'})$.
		\item В случае признания мастер-узлом допустимым включение записи $I$ в блок $b$, мастер-узел передаёт всем узлам из $B_{n^{'}}$ эту запись для проверки и включения в блок $b$. В противном случае запись $I$ отвергается без уведомления.
		\item Новая запись включается в блок до наступления момента $t^{'}$. Мастер-узел рассылает сообщение тем же узлам о фиксации блока $b$. Все узлы из $B_{n^{'}}$ вычисляют значение хеш-функции $H(b)$ равное, допустим, $h$. 
		\item	Каждый узел вычисляет электронную подпись:
			\begin{equation}
				s_{l}=sig(h),\;\begin{cases}
				l=1,\ldots,n{'}\\
				l\neq {\hat{k}}
				\end{cases}
			\end{equation}
		и передаёт её на узел  ${\hat{k}}$. 
			\item Мастер-узел ожидает электронные подписи время $\varDelta t$ после наступления момента $t{'}$. В момент $\varDelta t+t{'}$  на мастер-узле формируется кортеж
				\begin{equation}
						s_{b}=(s_{1},\ldots,s_{j}),\;1\leq j<n{'}
				\end{equation}
		Мастер-узел проверяет каждую подпись из (6) и подсчитывает число корректных подписей. Подписи некоторых узлов из $B_{n{'}}$  могут оказаться голосующими "против" или некорректными в том случае, когда в $B_{n{'}}$ окажется узел с некоторым номером$j_{\tilde{k}},\;1\leq\tilde{k}\leq n{'}$ который\\
		а) признает запись $I$ некорректной; 
\\
		б) в момент времени $\tilde{t}\in[t,t{'})$ имеет состояние блокчейна $\tilde{d}$, отличное от состояния $d$ для узла $j_{\hat{k}}$;
\\
		в) исказит запись $I$ при формировании блока блокчейна.
		\item 	Мастер-узел вычисляет количество корректных подписей ${\mu}$ и проверяет выполнение неравенства: 
				\begin{equation}
						\mu>\left[\frac{2}{3}n'\right]
				\end{equation}
		Если (7) не выполняется, то мастер-узел делает вывод, что консенсус не достигнут, в противном случае для блока $b$ составляется число:
				\begin{equation}
					b\parallel s_{k_{1}}\parallel\ldots\parallel s_{k_{\mu}},\;1\leq l<n{'},\;l=1,\ldots,\mu
				\end{equation}
		для которого вычисляется $H(b\parallel s_{1}\parallel\ldots\parallel s_{\mu})$  и $sig$, являющаяся электронной подписью узла с номером $j_{\hat{k}}$.
		
		\item Число
				\begin{equation}
						d{'}=b\parallel s_{1}\parallel\ldots\parallel s_{\mu}\parallel sig(H(b\parallel s_{1}\parallel\ldots\parallel s_{\mu}))
				\end{equation}
			назовём новым закрытым блоком, которое будем считать новым состоянием блокчейна $d{'}$, соответствующим моменту $t{'}$. Мастер-узел рассылает (9) всем узлам из множества $A_{N}$.
			\item 	На каждом узле из  $A_{N}$, предположим, с номером $1\leq m\leq N$, осуществляется проверка $s_{b}$  и $sig$ (9). Если проверка пройдена, то блок $b$ добавляется в блокчейн узла номер $m$ и блокчейн на узле переходит в состояние $d{'}=d(t{'})$. Если этот узел не получил указанных выше подписей для проверки в промежутке времени $[t{'}+\varDelta t,t{'}+\varDelta t+\lambda]$, где $\lambda$ \textendash{}  время задержки передачи информации, то узел с номером $m$ сочтёт консенсус недостигнутым и выберет новое множество $B_{n{''}}$ на основе старого состояния $d$, применяя (3).
			\end{enumerate}
	
		\section{Задание функции $f$ }\addcontentsline{toc}{section}{Function}
		Вычислим двойной хеш от (9), обозначив полученное число $\nu$. Построим псевдослучайную битовую последовательность вида:
				\begin{equation}
							\nu_{1}=H(H(d)),\;\nu_{2}=H(H(d+1)),\;\ldots
				\end{equation}
		\indent Получим следующую битовую запись:
		\begin{equation}
		R=\nu_{1}\parallel\nu_{2}\parallel\ldots
		\end{equation}
		\indent Битовую запись (11) разделяем последовательно без пропусков и перекрытий на кортежи по $r$  бит в каждом,  которые являются номерами узлов, образующих множество  $ B_{n{'}}$.  Эти номера обозначаются $j_{k}$, как указывалось выше, где $k=1,\ldots,n{'}\;,\;1\leq j_{k}\leq N$, причём $\hat{k}\triangleq1$, вследствие чего мастер-узлом всегда будет узел, номер которого сформируется первым. Если случайно повторится уже полученный номер $j_{k}$, то повторно полученное число пропускается.

\section{Выводы}\addcontentsline{toc}{section}{Resume}
	Предложенный алгоритм sdBFT по мнению авторов должен обладать более высоким быстродействием по сравнению с BFT алгоритмами. Изменяя мощность множества $B_{n{'}}$, можно будет управлять скоростью создания новых блоков, другими словами \textendash{} скоростью работы алгоритма.
	Потенциально большое число участников консенсуса усложняет предварительный сговор, когда группа голосующих узлов формирует новый блок, управляя составом блока по своему усмотрению, так как при следующем установлении консенсуса будет выбрано другое множество $B_{n{'}}$ голосующих узлов.
	Псевдослучайный выбор множества голосующих узлов $B_{n{'}}$ по мнению авторов не позволит оказать существенного влияния на выбор узлов при следующем голосовании.
	В дальнейшем авторы предполагают получить экспериментальное подтверждение теоретических положений, изложенных в этой статье, оценить скорость работы алгоритма sdBFT, исследовать возможные блокировки сети блокчейн.
	
	
\begin{thebibliography}{1}
	\bibitem{key-1} Satoshi Nakamoto (2009). ``Bitcoin: A Peer-to-Peer
	Electronic Cash System''. www.bitcoin.org .
	
	\bibitem{key-2} (2018.01.10). ``Transactions Speeds: How Do Cryptocurrencies
	Stack Up To Visa or PayPal?''. https://howmuch.net/articles/crypto-transaction-speeds-compared
	. 
	
	\bibitem{key-3} BitFury Group (2015.09.13). ``Proof of Stake versus
	Proof of Work''. http://bitfury.com/content/5-white-papers-research/pos-vs-pow-1.0.2.pdf
	.
	
	\bibitem{key-4} Andrew Miller, Yu Xia, Kyle Croman, Elaine Shi, Dawn
	Song (2016). ``The Honey Badger of BFT Protocols''. https://eprint.iacr.org/2016/199.pdf
	.
	
	\bibitem{key-5} Leslie Lamport, Robert Shostak, Marshall Pease (1982).
	``The Byzantine Generals Problem''. ACM Transactions on Programming
	Languages and Systems. T.4, 3: 382\textendash 401
. 
	
	\bibitem{key-6} National Institute of Standards and Technology. SHA-3
	Standard: Permutation-Based Hash and Extendable-Output Functions.
	https://nvlpubs.nist.gov/nistpubs/FIPS/NIST.FIPS.202.pdf
	.
		
	\bibitem{key-7} S.Josefsson, I.Liusvaara. ``Edwards-Curve Digital
	Signature Algorithm (EdDSA)''. IETF RFC. https://tools.ietf.org/html/rfc8032
	. 
\end{thebibliography}

\end{document}
