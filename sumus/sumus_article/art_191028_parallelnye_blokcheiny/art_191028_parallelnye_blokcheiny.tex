\documentclass{article}

\usepackage[english, russian]{babel} 
\usepackage[utf8]{inputenc} % Включаем поддержку UTF8

\usepackage[margin=2cm]{geometry}
\usepackage{indentfirst} % Красная строка после заголовка

\usepackage{amsmath} % Расширенное форматирование формул, в т.ч. "aligned"

\usepackage{graphicx,xcolor} % графика для svg
\graphicspath{{fig/}}


\title {Параллельные блокчейны.}
\author{noise@sumus.team}
\date{28 октября 2019 г.}

\begin{document}

\maketitle

% \begin{abstract} \end{abstract}

% \section{}

Пусть задано $m$ параллельных блокчейнов (сабчейнов) $B_{n_1}, \dots, B_{n_m} : B_n= \bigcup\limits_{j=1}^{m} B_{n_j} \subseteq A_N $.

Синхронным параллельным блокчейном назовём блокчейн, для которого в любой момент времени закрыто одинаковое количество блоков в каждом $B_{n_j}, j=1,\dots,m$.

Если блокчейн несинхронный, то для описания рассогласования сабчейнов и блокчейна в целом можно задать функцией $f_j(t), j=1,\dots,m  t \in [0,+\infty)$, где $f_j(t)$ равно количеству закрытых блоков в $B_{n_j}$ на момент $t$.

$f_j(t)$ --- кусочно-постоянная неубывающая функция вида $f_j(t) = K_j, t \in [t_{K,j}, t_{K+1,j})$, где $t_{K,j}$ --- момент закрытия блока номер $K_j$, $t_{K+1,j}$ --- момент закрытия блока номер $K_j+1$ в $B_{n_j}$;

Невязкой сабчейнов $B_{n_i}$, $B_{n_j}$ на интервале $[0, t^*]$ назовём (1) $I_{ij}= \int\limits_{0}^{t^*} |f_i(t) - f_j(t)| dt$.

Тогда невязкой блокчейна $B_n$ назовём (2) $I=\sum\limits_{j=2, i<j}^n I_{ij}$

Задача минимизации $I$ есть задача синхронизации блокчейна. Очевидно, что \\
$\inf{I}=0$ --- синхронный блокчейн \\
$\sup{I}=\int\limits_{0}^{t^*} |f_i(t)| dt$ --- параллельность не работает. \\

\paragraph{Рассмотрим синхронный блокчейн.} 
Пусть $T$ --- время от начала формирования одного блока до его закрытия, $T\le T_{fix}$ --- ограничение на время закрытия блока; $T$ является $const$ для всего блокчейна.

Будем сравнивать непараллельный блокчейн (нп-блокчейн) с параллельным (п-блокчейн). Когда в нп-блокчейне за время $T$ закрывается 1 блок, в параллельном блокчейне закрывается $m$ блоков, а значит среднее время закрытия одного блока в этом блокчейне есть $\dfrac{T}{m}$. Пусть $\widehat{\tau}$ -- время, затрачиваемое всеми узлами блокчейна (нп-блокчейн и п-блокчейн) на обработку одного сообщения о закрытии одного блока. В параллельном блокчейне из-за одновременного закрытия $m$ блоков время, затрачиваемое на обработку всеми узлами $m$ одновременно поступивших сообщений равно $m\widehat{\tau}$. При этом в нп-блокчейне требуется только $\widehat{\tau}$ времени для обработки единственного сообщения.

Время, затрачиваемое на соотнесение одной транзакции и кошелка узлами, участвующими в закрытии одного блока обозначим $\widetilde{\tau}$. Поскольку во всех блоках одинаковое количество транзакций, то время, затрачиваемое при закрытии 1 блока в нп-блокчейне и в п-блокчейне будет одинаковым и равным $K_T\cdot \widetilde{\tau}$.

Рассмотрим норму блокчейна как метрику, заданную на паре (``идеальный'' блокчейн, реальный блокчейн), где ``идеальный'' блокчейн в данном случае мгновенно выполняет любые операции (за время $o$). Рассмотрим пространство состояний блокчейна, где координата $x_1$ --- время на обработку сообщений о закрытии блоков всеми узлами, $x_2$ --- среднее время на закрытие одного блока,  $x_3$ --- время, затрачиваемое на соотнесение транзакций с кошельком. Для п-блокчейна $x_1  =m \cdot \widehat{\tau}$; $x_2 = \dfrac{T}{m}$; $x_3 = K_T \cdot \widetilde{\tau}$; для нп-блокчейна $x_1 = \widehat{\tau}$; $x_2 = T$; $x_3 = K_T \cdot \widetilde{\tau}$. П-блокчейн ``лучше'' нп-блокчейна если его норма меньше нормы нп-блокчейна, то есть для нормы $|x_1|+|x_2|+|x_3|$ получим $m \cdot \widehat{\tau} + \dfrac{T}{m} + K_T \cdot \widetilde{\tau} < \widehat{\tau} + T + K_T \cdot \widetilde{\tau}$ или \\
(3) $m \cdot \widehat{\tau} + \dfrac{T}{m} < \widehat{\tau} + T$;

Для решения неравенства (3) решим уравнение \\
(4) $m \cdot \widehat{\tau} + \dfrac{T}{m} = \widehat{\tau} + T$ или \\
(5) $\widehat{\tau} \cdot m^2 - (T + \widehat{\tau}) m + T = 0$. Решим это уравнение \\
$D = (T + \widehat{\tau})^2 - 4 T \widehat{\tau} = (T - \widehat{\tau})^2 \ge 0$; $m = \dfrac{T + \widehat{\tau} \pm |T - \widehat{\tau}|}{2 \widehat{\tau}}$.

Условия ``физически'' реального решения: \\
(1) $D \ge 0$; (2) $m \ge 1$; из первого условия получим (3) $T \ge \widehat{\tau}$, откуда получим корни квадратного уравнения \\
$m_1 = 1$; $m_2 = \dfrac{T}{\widehat{\tau}} \ge 1$; $m \in [m_1, m_2]$ --- интервал допустимых значений $m$.

Найдём $m_{opt}$, обеспечивающее $\min$ нормы п-блокчейна: \\
(5) $m_{opt} = \arg \min \bigg( m \widehat{\tau} + \dfrac{T}{m} \bigg), m \in \bigg[ 1, \dfrac{T}{\widehat{\tau}} \bigg] $ \\
$\bigg(m \widehat{\tau} + \dfrac{T}{m} \bigg)' = \widehat{\tau} - \dfrac{T}{m^2} = 0$ или \\
$m_{opt} = \bigg(\dfrac{T}{\widehat{\tau}} \bigg) ^{\dfrac{1}{2}}$.
Заметим, что в $m_{opt} : m \widehat{\tau} = \dfrac{T}{m}$. \\
$m_{opt} \in [m_1, m_2]$.

\paragraph{Дополнительные условия.}
Дополнительной причиной рассогласования п-блокчейна может стать ``уход в раунды'' одного или нескольких сабчейнов (из-за ухода в раунды при попытке принятия транзакции). Это может привести к уменьшению количества работающих сабчейнов на интервале $[0, t^*]$.

Пусть $l$ --- число сабчейнов, которые ушли в ``раунды'' на интервале $[0, t^*]$. Если известно распределение $l^*$ как случайной величины, например, нормальное распределение с МО $l^*$, то надо учитывать, что количество работающих сабчейнов в среднем на большом $t^*$ будет меньше $m - l^*$.

Пусть множество $\mathbf{B}$ --- множество блоков, закрытых на момент времени $t^*$, $B_n$ (или $A_N$) --- множество узлов, принимавших участие в закрытии блоков. Если между  и  может быть установлено взаимно однозначное соответствие (биекция) в котором каждому блоку поставлен в соответствие узел, его закрывший (без консенсуса), то такой блокчейн назовём полностью децентрализованным.

Если все блоки закрыты одним узлом (мастер-узлом), то это полностью централизованный блокчейн.

Будем считать, что полностью централизованному блокчейну соответствует значение коэффициента $K = 1$, где \dots

$\alpha_1$ --- мощность множества узлов $B_n$ \dots

$1-\frac{\alpha_1}{\alpha_2} = \frac{\alpha_2 - \alpha_1}{\alpha_2}$;

$r = \frac{1}{\alpha_2}$ --- полностью централизованный блокчейн.

$\alpha_2$ --- мощность $\mathbf{B}$

$\alpha_1$ --- количество узлов, закрывших меньше $\alpha_2$ блоков в момент времени $t^*$

если $\alpha_1 = 0$, $K = 1$ --- полностью централизованный блокчейн.

\end{document}
