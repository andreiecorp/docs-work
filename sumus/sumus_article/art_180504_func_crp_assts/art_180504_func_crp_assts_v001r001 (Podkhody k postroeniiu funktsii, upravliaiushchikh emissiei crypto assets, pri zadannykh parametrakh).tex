\documentclass[a4paper, 12pt]{article}
\usepackage[utf8]{inputenc}
\usepackage[T2A]{fontenc}
\usepackage[english, russian]{babel} 
\usepackage{amscd, amsmath, amssymb, amsthm, enumerate, indentfirst, longtable, ifthen} %% addlibrary rom mathematic 
\usepackage{ragged2e}
\usepackage{microtype}
\usepackage{indentfirst}
\usepackage{setspace}
\usepackage{pgfplots} %graf create
\setstretch{1.15} % interval 
%% PGFPlots setting
\pgfplotsset{compat=1.9}

 %%picture 
\RequirePackage{caption}
\DeclareCaptionLabelSeparator{defffis}{. }
\captionsetup{justification=centering,labelsep=defffis}
\usepackage{graphicx}
\graphicspath{{pict/}}
\DeclareGraphicsExtensions{.png,}
 %%end picture 
\usepackage{geometry}
\geometry{
 	tmargin=15mm,         % Up 
 	bmargin=15mm,         % Down 
 	lmargin=15mm,         % left поле
 	rmargin=15mm,         % right  
}

%% new command 
\newcommand{\jj}{\righthyphenmin=20 \justifying}
%opening
\date{4 мая, 2018}
\title{Подходы к построению  функций, управляющих эмиссией crypto assets,  при заданных параметрах }
\author{a@sumus.team, rr@sumus.team}

\begin{document}

\maketitle

\begin{abstract} 
	Постоянно возникает вопрос как для систем в которых обращаются crypto assets проводить их эмиссию. Существуют различные подходы, кто-то наделяет один центр полномочиями по эмиссии crypto assets
\end{abstract}

\section{Введение}\addcontentsline{toc}{section}{Введение}
Разобьём  вопрос о распределении $fee$ между узлами на две части\\


Выбор кортежа очередных нод для участия в консенсусе, может содержать и белые и чёрные НОДы, что является простой схемой случайных опытов, которая называется -  схема Бернулли. В общем случае, вероятность элементарного события содержащего $\mu$ успехов составляет

			\begin{equation}				
			           \label{equation:bernuli} 
						P\{\mu=m\} = P_n(m) = C_{n}^{m} \cdot n^{m} \cdot q^{n-m}						
			\end{equation} где $m = \overline{0,n}$; $C_n^m$ -- это количество комбинаций (сочетаний) $m$ различных элементов без учёта порядка их появления из $n$ элементов; $p$ -- вероятность ``успеха'' в отдельном испытании; $q$ -- вероятность неуспеха в отдельном испытании, т.е. $q = 1 - p$ . Числа $P_n(m)$ называются биномиальными вероятностями, а формула (\ref{equation:bernuli}) называется формулой Бернулли.

			Одно из следствий схемы Бернулли, это определение вероятности события $A$, которое произойдёт в $n$ испытаниях не менее $r$ раз, но не более $k$ раз, равна
			\begin{equation}\label{equation:sum_ber}
				P(A) = \sum _{m=r}^k P_n(m)			
			\end{equation}

			Чтобы определить вероятность того, что ``белые'' НОДы не смогут наложить ``вето'' на блок сформированный ``черными'' нодами проведём расчёт. Для этого определим для нашей задачи исходные данные.
			\begin{enumerate}
				\item 	$n=100$ -- размер кортежа НОД участвующих в консенсусе, т.е. количество испытаний; 
				\item 	$m$ -- количество ``белых'' НОД;
				\item   общее количество НОД в системе $10 000$, количество ``белых'' НОД $4 999$, а ``чёрных'' НОД $5 001$;
				\item  соответственно вероятность $p=\frac{4999}{10000}=0.4999$, а $q=\frac{5001}{10000}=0.5001$.
			\end{enumerate}
			Следовательно
				\begin{equation}\label{equation:sum_ras}
			P_{100}(m\leq[1/3 \cdot 100]) = \sum _{m=0}^{33} C_{100}^m \cdot 0.4999^{100-m} \cdot 0.5001^m \approx \\
			0.00088893...
			\end{equation}
			Т.е. вероятность наступления события при котором белые НОДы не смогут заблокировать создание ложного блока составит $\approx 9 \cdot 10^{-4}$.
			
			Определим теперь вероятность создания "легального" блока белыми нодами
			\begin{equation}\label{equation:sum_ras2}
					P_{100}(m > [2/3 \cdot 100]) = \sum _{m=67}^{100} C_{100}^m \cdot 0.4999^{100-m} \cdot 0.5001^m \approx \\
					0.00043998...
			\end{equation}
				Т.е. вероятность наступления события при котором белые НОДы  смогут создать "легальный" блок составит $\approx 4 \cdot 10^{-4}$.
			
			
			
			Мы исходим из того, что задача достижения консенсуса в первом приближении сводится к получению участниками распределенной системы согласованного решения в случае, если некоторое их количество не приняло участие в согласовании решения. Это может произойти по следующим причинам:
			
			\begin{enumerate}
				\item 	Ошибка при передаче сообщения о принятии решения одним из участников.
				\item 	Слишком медленная передача сообщения о принятии решения одним из участников.
				\item Сбой в работе участника в системе.
				\item Введение в заблуждение при принятии решения в системе, как умышленное, так и неумышленное.
			\end{enumerate}
			
			Если рассматривать причины 1-3 неучастия в выработке консенсуса, то для принятия решения достаточно чтобы выполнялось условие $n>m+1$ \cite{key-5}, где $m$ \textendash{} количество участников, которые не участвовали в консенсусе, а $n$ \textendash{} количество участников принявших решение. В случае появления в системе участников, не участвующих в консенсусе по четвертой причине задача сводится к задаче византийских генералов, которая имеет решение, когда
			\begin{equation}
			n>3\cdot m
			\end{equation}
			
			Если вместо $n$ генералов рассмотреть $n$ узлов блокчейна, участвующих в выработке консенсуса, то очевидно, что эта задача подобна задаче византийских генералов. Для решения проблемы роста времени достижения консенсуса предлагается из конечного множества узлов $B_{n}$ ,мощность этого множества $\overline{B}_{n}=n$, выделять подмножество $\overline{B}_{n^{'}}\;(B_{n^{'}}=n^{'})$ и исходя из предположения о равномерности распределения свойств узлов во всей сети блокчейн решать задачу не на $B_{n}$, а на $B_{n^{'}}$ при $n\gg n^{'}$. Общее количество узлов во всей сети блокчейн положим равным $N$. Обозначим множество этих узлов $A_{N}$.
										
\begin{thebibliography}{1}
	\bibitem{key-1} Satoshi Nakamoto (2009). ``Bitcoin: A Peer-to-Peer
	Electronic Cash System''. www.bitcoin.org .
	

	. 
\end{thebibliography}
\end{document}
