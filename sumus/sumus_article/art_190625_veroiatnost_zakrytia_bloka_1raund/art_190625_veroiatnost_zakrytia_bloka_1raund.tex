\documentclass{article}

\usepackage[utf8]{inputenc} % Включаем поддержку UTF8
\usepackage[russian]{babel} % [english, russian]

\usepackage[margin=2cm]{geometry}
\usepackage{indentfirst} % Красная строка после заголовка

\usepackage{enumitem} % расширение для enumerate
\makeatletter
\AddEnumerateCounter{\asbuk}{\russian@alph}{щ}
\makeatother

\usepackage{amsmath} % Расширенное форматирование формул, в т.ч. "aligned"

\title{Вероятность закрытия блока за 1 раунд, если из $n$ узлов $m$ отключены.}
\author{noise@sumus.team}
\date{25 июня 2019 г.}

\begin{document}

\maketitle

\def\np{n^\prime}

\begin{abstract}
	\dots \\
\end{abstract}

\paragraph*{Допущение}

\begin{enumerate}[label=\asbuk*),ref=\asbuk*]
	\item Вероятность закрытия блока при $m=0$ равна 1.
	\label{enum:1a}
	\item Эскорт выбирается 1 раз для закрытия 1 блока.
\end{enumerate}

\paragraph*{Следствие 1}

Если допущение \ref{enum:1a} верно, то если при выборе из $B_n$ эскорта $B_{\np}$ все $\np$ узлов подряд оказались неотключёнными, то консенсус достигается за 1 раунд.

%
Вероятность $p=\frac{n-m}{n}$ -- того, что при $m$ отключенных узлах за одно испытание будет выбран подключённый узел. $q=1-p=\frac{m}{n}$ -- вероятность противоположного исхода.

%
Это эквивалентно схеме испытаний без ``возврата''. Так, у нас в урне $n$ шаров из которых $m$ чёрных и $n-m$ белых. Какова вероятность того, что можно вытащить за $\np$ попыток $\np$ белых шаров (то есть ``подряд''), если после каждой попытки шар не возвращается в урну. Здесь подразумевается, что ``набор'' одного эскорта $B_{\np}$ это вытаскивание подряд $\np$ шаров из урны 

%
В  $B_{\np}$ может быть $< \left[\frac\np{3}\right]$ неподключённых узлов (по сравнению со следствием 1).

\paragraph*{Дополнение. }
$B_n$ статистически подобно $A_n$

%
Вероятность того, что за $\np$ попыток будет выбрано из $n$ узлов $\np$ неотключённых равна
\begin{equation*}
	p_{\np}=\frac{\np!}{\np!(\np-\np)!}
	\cdot p^{\np}
	\cdot (1-p)^{\np-\np} =
\end{equation*}

\begin{equation*}
	C_{\np}^{\np}
\end{equation*}

\begin{equation}
	p_{\np}=\left(\frac{n-m}{n}\right)^{\np}
	\label{equation:p_nprim}
\end{equation}
-- с ``возвратом''.

%
На самом деле -- без ``возврата''. Тогда
\begin{equation}
	p_{\np}
	=\frac{C_{n-m}^{\np}\cdot C_{m}^{\np-\np}}{C_{n}^{\np}}
	=\frac{\frac{(n-m)!}{n! \:?\: (n-m-\np)!}\cdot\frac{m!}{0!m!}}{\frac{n!}{\np!(n-\np)!}}
	=\frac{(n-m)!(n-\np)!}{\np(n-m-\np)!}
	\label{equation:p_nprim2}
\end{equation}

%
Если число попыток $k\ge \np$, ($k\le \np+m$ -- физическое ограничение) то формула (\ref{equation:p_nprim2}) примет вид

\begin{equation}
	p_{k}=\frac{C_{n-m}^{\np} \cdot C_{m}^{k-\np}} {C_{n}^{k}}
	=\frac{
		\frac{(n-m)!}{n!(n-m-\np)!}
		\cdot
		\frac{m!}{(k-\np!)(m-k+\np)!}
	}{
		\frac{n!}{\np!(n-\np)!}}
	=\frac{
		(n-m)!(n-\np)!m!
	}{
		n!(n-m-\np)!(k-\np)!(m-k+\np)!}
	\label{equation:p_k}
\end{equation}

%
Вероятность того, что без ``возврата'' при $m$ отключённых узлах за $k$ попыток в $B_{\np}$ попадёт $\hat{n}$ подключённых узлов $k\le \widehat{n}+m; \: k\ge \np$

\begin{equation*}
	p_k=\frac{C_{n-m}^{\widehat{n}}\cdot C_m^{k-\widehat{n}}}{C_n^k}
\end{equation*}

\section{Вторая постановка задачи о вероятности закрытия блока за 1 раунд, когда из $n$ узлов в $B_n$ отключены $m$ узлов.}

\begin{enumerate}[label=\asbuk*),ref=\asbuk*]
	\item Вероятность закрытия блока за 1 раунд равна 1 только при $m=0$.
	\item Эскорт $B_{\np}$ выбирается 1 раз для закрытия 1 блока.
	\item 1 блок может быть закрыт за 1 раунд, если в эскорт  первым попал неотключённый узел, и в целом за $\np$ попыток в эскорт попало $n^*+1$ неотключённых узлов : $n_1\le n^*+1\le n_2$, включая мастер-узел.
\end{enumerate}

\section{Алгоритм}

\paragraph{1.}
Вероятность того, что с первой попытки в эскорт попадёт неотключённый узел и станет мастер-узлом равна $p_1=\frac{n-m}{n}$.

\paragraph{2.}
Вероятность того, что из оставшихся $n-1$ узлов за оставшиеся $\np-1$ попытку будет выбрано $n^*\le \np-1$ неоключённых узлов.

\begin{equation*}
	p_{n^*}
	=\frac{C_{n-m}^{n^*}\cdot C_m^{\np-n^*-1}}{C_n^{\np-1}}
	=\frac{
	(n-m)!(n-n^*)!m!}{n!(n-m-n^*)!(\np-n^*-1)!(m-\np+n^*+1)!}
\end{equation*}

\begin{equation*}
	\np-1\le n^*+m
\end{equation*}

\begin{equation*}
	n^*>\left\lceil \frac{3}{4} \np \right\rceil - 1
\end{equation*}
$\lceil \bullet \rceil$ ``Потолок'' -- округляется в сторону ближайшего большего целого числа.

\paragraph{3.}
Вероятность того, что в $B_{\np}$ попадёт с первой попытки неотключённый узел, с остальных $\np-1$ попыток $n^*$ неотключённых узлов, есть произведение $p_1$ на $p_{n^*}$
\begin{equation}
	p=\frac{n-m}{n}\cdot\frac{(n-m)!(n-n^*)!m!}{n!(n-m-n^*)!(\np-n^*-1)!(m-\np+n^*+1)!}
	\label{equation:p}
\end{equation}

\paragraph{2'.}
Если вместо п.2 алгоритма потребовать, чтобы за оставшуюся $\np-1$ попытку (опыт) из $n-1$ узла будет выбрано $\lceil \frac{3}{4} \np \rceil \le n^* \le \np-1$ ($n_1 = \lceil \frac{3}{4} \np \rceil; \; n_2 = \np-1$ ) неотключённых узлов, то искомая вероятность есть сумма $p_{n^*}$ при $n^* = \lceil \frac{3}{4} \np \rceil, \dots, \np - 1$ :
\begin{equation*}
p_{n_1\le n^*\le n_2} = \sum_{n^*=n_1}^{n_2}p_{n^*}
\end{equation*}

%
Тогда вместо формулы (\ref{equation:p}) будет получено:
\begin{equation}
	p=\frac{n-m}{n}
	\cdot
	\sum_{n^*=\lceil \frac{3}{4} \np \rceil}^{\np-1}
	\frac{(n-m)!(n-n^*)!m!}{n!(n-m-n^*)!(\np-n^*-1)!(m-\np+n^*+1)!}
	\label{equation:p2}
\end{equation}


\end{document}
