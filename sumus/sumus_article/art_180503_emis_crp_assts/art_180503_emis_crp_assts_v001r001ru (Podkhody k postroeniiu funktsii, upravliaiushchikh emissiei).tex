\documentclass[a4paper, 12pt]{article}
\usepackage[utf8]{inputenc}
\usepackage[T2A]{fontenc}
\usepackage[english, russian]{babel} 
\usepackage{amscd, amsmath, amssymb, amsthm, enumerate, indentfirst, longtable, ifthen} %% addlibrary rom mathematic 
\usepackage{ragged2e}
\usepackage{microtype}
\usepackage{indentfirst}
\usepackage{setspace}
\usepackage{pgfplots} %graf create
\setstretch{1.15} % interval 
%% PGFPlots setting
\pgfplotsset{compat=1.9}

 %%picture 
\RequirePackage{caption}
\DeclareCaptionLabelSeparator{defffis}{. }
\captionsetup{justification=centering,labelsep=defffis}
\usepackage{graphicx}
\graphicspath{{pict/}}
\DeclareGraphicsExtensions{.png,}
 %%end picture 
\usepackage{geometry}
\geometry{
 	tmargin=15mm,         % Up 
 	bmargin=15mm,         % Down 
 	lmargin=15mm,         % left поле
 	rmargin=15mm,         % right  
}

%% new command 
\newcommand{\jj}{\righthyphenmin=20 \justifying}
%opening
\title{Подходы к построению  функций, управляющих эмиссией crypto assets,  при заданных исходных параметрах }
\author{noise@sumus.team, k@sumus.team, a@sumus.team}

\begin{document}

\maketitle

\begin{abstract} 
	Постоянно возникает вопрос, как для систем в которых обращаются crypto assets проводить их эмиссию. Существуют различные подходы, кто-то наделяет один центр полномочиями по эмиссии crypto assets, а кто-то реализует функцию как алгоритм.
\end{abstract}

\section{Введение}\addcontentsline{toc}{section}{Введение}
Разобьём  вопрос о распределении $fee$ между узлами на две части\\

\underline{Часть 1.} 

Если все узлы из $B_n$ постоянно присутствуют в сети, то они равноправны в том смысле, что вероятность получения $\forall$  узлом в произвольным момент $t$ закрытия блока вознаграждения в интервале величина, которого зафиксирована (достигнута) на данный момент, зависит только от величины интервала и количества узлов $n$ . \\

					Будем считать $fee$ функцией времени $t$, где в некоторый момент $t_k$ соответствующей проведению отдельной транзакции $fee(t_k)$ принимает значение, равное fee, ``начисленному'' за эту транзакцию. Пусть $T$ - счётное множество значений всех $t_k$, в которые осуществляются транзакции, $k=1,2,...$ Для общности будем считать, что множество всех $\mathbb{N}$  есть множество значений этой функции обозначим как $\Psi$. Таким образом, $fee(t_k)$ есть решетчатая неотрицательная функция, заданная на счётном множестве $T$.  Поскольку fee меняется по закону, который можно установить только статистическими методами, то положим, что $fee(t_k)$ есть решетчатая случайна функция, или, другими словами, дискретный случайны процесс. Вообще говоря, этот процесс может быть нестационарным, но при этом следует допустить его эргодичность и слабую коррелированность. Также будем считать, что у этого процесса есть математическое ожидание и дисперсия. Пусть плотность распределения этого процесса есть  $f(fee,t)$, где $t \in T$, $fee \in \Psi$.   $\Phi$ от $f$ не менее чем, непрерывна по $fee$, т.е. $fee(t_k)$ при фиксированном $t_k$ является непрерывной случайно величиной. 

Пусть также имеет место независимости от процесса закрытия блоков, который в данном случае можно представить как дискретный случайный процесс $g(t_m)$ значениями которого являются номера мастер-узлов $j_{\hat{k}}$ определяемые в моменты  $t_{m^*} \in [t_{m-1}', t_{m}')$, $m=1,2,3\dots$ , где $t_{m}'$ - момент закрытия блока номер $m$, $t_0'$ - начальный момент работы сети. Процесс $g(t_m^*)$ - стационарный с равномерным распределением дискретной случайно величины $j_k$, $k=1,\dots,n$, $1 \leq j_k \leq N $ с плотностью вероятности 
		\begin{equation}
		   f_g(k)=\sum_{k=1}^{n} \frac{1}{n}\sigma(k-k_{\nu})
		\end{equation}
независимый от $fee(t_k)$. Счётное множество всех значений $t_m^*$ обозначим $T^*$. 

В момент времени $t_m$ закрытия блока номер $m$ мастер-узел  с номером $j_{\hat{k}}$ получает всю накопленную к этому моменту $\Delta fee_{\Sigma}$, которую будем считать равной
		\begin{equation}
		\Delta fee_{\Sigma} (t_{m}^{'}) = \sum_{k=k_{m-1}}^{k_m} fee(t_k)
		\end{equation}
где $k_{m-1}$ - номер момента $t_{k_{m-1}}$, ближайшего к $t_{m-1}^{'}$ справа, а $k_m$ - номер момента $t_{k_m}$, ближайшего к $t_{m}^{'}$ слева.

					Поскольку для $\forall_{k}, m : t_k - t_{k-1} \ll t_{m}^{'}-t_{m-1}{'}$, то дискретная случайная функция $\Delta fee_{\Sigma} (t_{m}^{'}) $ является суммой большого числа случайных величин и согласно закону больших чисел (здесь считаются выполненными условия теоремы Маркова и Чебышева для дисперсии), имеет распределение близкое к нормальному, т.е.:
		\begin{equation}
		F(\Delta fee_{\Sigma} < a, t))  \approx \Phi (\Delta fee_{\Sigma} < a, t))
		\end{equation}
					и является при произвольных, но подобных $f(fee,е)$  дискретным случайным процессом с неотрицательными значениями. 

					Таким образом, каждый из $n$ узлов может в момент $t_{m}^{*}$ стать мастер-узлом с вероятностью $\displaystyle \frac{1}{n}$ и в соответствующий  момент $t_{m}^{'}$ получить $\Delta fee_{\Sigma} (t_{m}^{'})$ например в интервале  $(M_{\Delta fee_{\Sigma}}-3\sigma_{\Delta fee_{\Sigma}}  ,   M_{\Delta fee_{\Sigma}}+3\sigma_{\Delta fee_{\Sigma}})$ с вероятностью $\approx{0,997}$, где $M_{\Delta fee_{\Sigma}}$ и $\sigma_{\Delta fee_{\Sigma}}$ - математическое ожидание и среднеквадратическое отклонение случайного процесса в момент времени  $t_{m}^{'}$.

					В результате, в момент  $t_{m}^{'}$  каждый из $n$ узлов получает указанное вознаграждение с вероятностью $ \displaystyle\frac{0,997}{n}$. 
					
					Приведённые выше результаты справедливы при достаточно длительных реализациях случайных процессов, т.е. при $m \gg n$ и их эргодичности.

\underline{Часть 2.}

Если не все узлы из $B_n$ постоянно присутствуют в сети, то это означает, что есть некоторое фиксированное подмножество узлов $B_{\widetilde{n}} \subseteq B_n $, $\widetilde{n} \leqslant  n $ такое, что для $\forall$ узла из $B_{\widetilde{n}} $ справедливо следующее: пусть этот узел  выбирался мастер-узлом $\bar{m}<m$ раз на отрезке $t \in [t_{0}', t_{m}^{*})$, но принял задачу по закрытию блока только $\widetilde{m} $ раз $(\bar{m} < \widetilde{m} )$, то за закрытие текущего блока, выбор эскорта которого осуществлён в момент $t_{m}^{*}$, этот мастер узел получит вознаграждение 
			\begin{equation}
			\gamma =  \frac{\widetilde{m}}{\bar{m}} \Delta fee_{\Sigma} (t_{m}^{'})  
				\label{equation:for1} 
			\end{equation}
остаток
			\begin{equation}\label{equation:for2}
			\frac{\bar{m}-\widetilde{m}}{\bar{m}} \Delta fee_{\Sigma} (t_{m}^{'})  
		     \tag{\ref{equation:for1}$a$}
			\end{equation}
суммируется в дальнейшем с $\Delta fee_{\Sigma} (t_{m+1}^{'})$. Ленивых НОД не может быть больше $\frac{n}{3}$. Можно сказать, что в данном разделе предложено правило мотивации постоянного подключения узла к сети.

В силу независимости процесса включения/отключения узла в сети от других случайных процессов, упомянутых выше и естественном предположении о стационарности и равномерности распределения номеров $\widetilde{n}$ отключаемых узлов, можно утверждать, что введение ``мотивирующих'' коэффициентов  $\displaystyle M=\frac{\widetilde{m}}{\bar{m}}$,  $\displaystyle 1-M=1-\frac{\widetilde{m}}{\bar{m}}$ /не понял эту формулу/и перенос остатков не приведёт к изменению классификации случайных процессов, используемых в этой задаче. 

Вероятность того, что в произвольный момент $t_{m}^*$ определения нового мастер-узла им станет непостоянно подключенный узел из $B_{\widetilde{n}}$  есть $   \widetilde{p}=\displaystyle\frac{\widetilde{n}}{n}$. Тогда в момент $t_{m}^{'}$  закрытия блока номер $m$ с данным мастер-узлом этот узел получит вознаграждение $\displaystyle \frac{\widetilde{m}}{m}\Delta fee_{\Sigma}(t_{m}^{'})$. Если следующий мастер-узел не принадлежит $B_{\widetilde{n}}$, то в момент $t_{m+1}^{'}$ он получит вознаграждение 
					\begin{equation}\label{equation:sumbonus}
							\Delta fee_{\Sigma}(t_{m+1}^{'}) + \frac{\bar{m}-\widetilde{m}}{m} \cdot \Delta fee_{\Sigma}(t_{m}^{'})
					\end{equation}
					
В момент $t_{m+2}^{'}$ новый мастер-узел, если он вновь не принадлежит  $B_{\widetilde{n}}$, то получит за закрытие $m+2$ блока $	\Delta fee_{\Sigma}(t_{m+2}^{'})$ уже без ``надбавок'' и в этом случае перенос выплаты $\displaystyle \frac{\bar{m}-\widetilde{m}}{m} \cdot \Delta fee_{\Sigma}(t_{m}^{'}) $ станет надбавкой соответствующему мастер-узлу, стимулируя стремления всех узлов из $B_{{n}}$ быть постоянно подключёнными и это не приведёт к неконтролируемому росту ``надбавок''  к $\Delta fee$. Если новый мастер-узел $\in B_{\widetilde{n}}$, то в момент $t_{m+2}^{'}$ повториться ситуация сложившаяся в момент $t_{m}^{'}$ и это значит, что если появление мастер-узлов из  $B_{\widetilde{n}}$ идёт не подряд, а хотя бы ``через один'', то накопленные премии не происходит. 

Пусть на отрезке  $[t_{0}^{'}, t_{m}^{*}]$ из  $m$ мастер-узлов  $\hat{m}$ оказались из $B_{\widetilde{n}}$   (с возможными повторными назначениями одних и тех же узлов мастер-узлами). Вероятность этого есть биноминальное распределение
					\begin{equation}\label{equation:bernuli}
							P_{\widetilde{m},m} = C_{m}^{\hat{m}}\cdot \widetilde{p}^{\ \hat{m}} \cdot (1-\widetilde{p})^{m-\widetilde{m}}
					\end{equation}
			Вероятность события ``из $m$ опытов $\hat{m}$ раз появляется номер узла из $B_{\widetilde{n}}$'' равна
					\begin{equation}\label{equation:probability}
		 					\widetilde{p}^{\ \hat{m}} \cdot (1-\widetilde{p})^{m-\widetilde{m}}
					\end{equation}
			если выбирать только один набор $t_{k_{i}}^{*}$, $i=1,\dots,\widetilde{m}$, $1 \leq k_j \leq m$, в котором появляются мастер-узлы из $B_{\widetilde{n}}$ . Наиболее ``критичным'' с точки зрения накопления надбавки является вариант, когда закрывается $\hat{m}$ блоков подряд, для каждого из которых мастер-узел принадлежал  $B_{\widetilde{n}}$. Вероятность этого равна (\ref{equation:probability}). Поскольку каждый узел из  $B_{\widetilde{n}}$ включается и отключается по-своему, то для оценки накопления вознаграждения за счёт премии
					$\displaystyle 
											\max_{i=1,\dots,\hat{m}}
														\Bigl\{
																	\frac
																				{\bar{m}{_{k_{i}}}-\widetilde{m}_{k_{i}}}
																				{\bar{m}{_{k_i}}}
														\Bigr\}
											=M_{\max}<1
					$,  соответственно 
					$\displaystyle 
								M_{\min}=
								\min_{i=1,\dots,\hat{m}}
												\Bigl\{
																\frac
																{\widetilde{m}_{k_{i}}}
																{\bar{m}{_{k_i}}}
																=M_{k_{i}}
												\Bigr\}
								<1
					$. Очевидно, что для $M_{\max}$ и $M_{\min}$ получаются при одном и том же $M_{k_{i}}$. Т.е. $M_{\max}=1-M_{\min}$.
					
					Положим $\displaystyle
												\overline{\Delta fee_{\Sigma}}=
												\max_{i=1,\dots,\hat{m}}
												\Delta fee_{\Sigma}(t_{k_{i}}^{'})
									$ 
										Оценка суммы вознаграждения в целом за закрытие $\hat{m}$ блоков в этом случае $\hat{m} \cdot \Delta fee $. Оценка суммы выплаченных вознаграждений есть $\hat{m} \cdot  M_{\min} \cdot \Delta fee $  (учитывая невыплату надбавки). Верхняя оценка суммы невыплаченного вознаграждения за $\hat{m}$ шагов есть 
										$
													\hat{m} \cdot  (1-M_{\min}) \overline{\cdot \Delta fee} 
										$ 
										При достаточно большом $\hat{m}$ в момент $t_{k{\hat{m}+1}}$ первый мастер узел, не принадлежащий $B_{\widetilde{n}}$ помимо  $\Delta fee_{\Sigma}(t_{k{\hat{m}+1}})$ получит надбавку  
										$
										\hat{m} \cdot  (1-M_{\min}) \overline{\cdot \Delta fee} 
										$ которая может оказаться чрезмерно большой.
									
			Если не использовать оценки надбавок и вознаграждений, которые введены для сокращения выкладок, и сохранить точные значения, то приведённые выше выводы останутся справедливыми. Действительно, сумма выплаченных вознаграждений $S$ есть
									
			\begin{equation}\label{equation:div1} 
						S=\sum_{i=1}^{\hat{m}}  \Delta fee_{\Sigma}(t_{k_{i}}^{'})\cdot M_{i} 
			\end{equation}
							а сумма $S_M$ невыплаченных вознаграждений за закрытие $\hat{m}$ блоков будет равна: 
			\begin{equation}\label{equation:div2}
						S_{M}=\sum_{i=1}^{\hat{m}}  \Delta fee_{\Sigma}(t_{k_{i}}^{'}) - S = \sum_{i=1}^{\hat{m}}  (1-M_i) \Delta fee_{\Sigma}(t_{k_{i}}^{'}) 
							\tag{\ref{equation:div1}$a$}
			\end{equation}					
					Одновременно $S_M$ является надбавкой, получаемой мастер узлом $\notin  B_{\widetilde{n}}$, следующим сразу за $\hat{m}$ узлами из $B_{\widetilde{n}}$.
					
					Так как описанная выше ситуация относится к числу маловероятных, то она не может привести в целом к заметному нарушению правила мотивации. Действительно если например $\widetilde{p}=0.1$, $m=10^2$, $\hat{m}=5$, то
						\begin{equation}\label{equation:calculations1} 
									 \widetilde{p}^{\ \hat{m}} \cdot (1-\widetilde{p})^{m-\widetilde{m}}=10^{-5} \cdot (0.9)^{95} 
						\end{equation}
					 поскольку $0.9^{95} \ll 10^{-2}$, то вероятность такого события $\ll 10^{-7}$.
					 
					 Однако для того, чтобы даже в таких случаях не было сомнений в получении пропорциональной надбавки  узлом, следующим за $\hat{m}$ и не принадлежащим 	$B_{\widetilde{n}}$, следуюет ввести функцию, позволяющую более дифференцированно подходить к ``штрафованию'' узлов из $B_{\widetilde{n}}$. Например, ппр идостаточно большом $M_i$ (ненамного меньше $1$) снижение выплаты таким узлам не происходит.
					 
					 Обозначим эту функцию $\varphi(M), M \in [0,1], \varphi \in [0,1]$. Все $M_m$ принадлежат ее множеству определения. Тогда вознаграждение мастер-узла за закрытие $m$-го блока в момент $t_{m}^{'}$ будет равно
					 \begin{equation}\label{equation:penalty} 
					 				\gamma_{\varphi} = \varphi(M) \cdot \Delta fee_{\Sigma}(t_{m}^{'}) 
					 \end{equation}				
					Функция $\varphi(M) $ является неубывающей. Примером такой функции может быть следующая функция 
					 \begin{equation}\label{equation:penalty_arc} 
									\varphi(M) = \frac{\arctan(M-M_1)+\arctan(M_1)}{\arctan(1-M_1)+\arctan(M_1)}, M_1 \in[0,1]
					\end{equation}		
					графики функции при разных значениях $M_1$ показаны на рисунке \ref{tikzpicture:gr1}  
				
					\begin{center}
							\begin{tikzpicture}
								\label{tikzpicture:gr1}
								\begin{axis}[
								title = $\varphi(M)$,
								xlabel = {$M$},
								ylabel = {$\varphi$},
								minor tick num = 1,
								domain=0:1,
								xmin=0, xmax=1, ymin=0, ymax=1,
							%%	mark=*,
								grid=major,
					%%			extra M ticks=0.9
								legend style = {
															at={(0.03,0.97)}, anchor=north west
														}
								]
								\addplot [red]{
														( atan(x-0.99)+atan(0.99)) /
											 			( atan(1-0.99)+atan(0.99)) 
													}; \addlegendentry{$M_1=0.99$}
								\addplot [blue] {
														( atan(x-0.01)+atan(0.01)) /
														( atan(1-0.01)+atan(0.01)) 
												};\addlegendentry{$M_1=0.01$}
							%%	\addplot coordinates {(0,0) (0.5,0.2) (1,1)};
							  %%  \addplot coordinates {(0.5,0.2)};				
								\end{axis}
							\end{tikzpicture}
							\captionof{figure}{График гладкой функции $\varphi$}
					\end{center}				
	
			Эта функция является гладкой, возможность управления уровнем выплат отражена одним параметром $M_1$ и при этом невозможно задать интервал, значений $M$, на котором при достаточно больших  $M$ не происходит штрафование соответствующего этому значению мастер-узла. 
			
			Следующий вариант функции $\varphi(M)$ имеет значительные преимущества перед \ref{equation:penalty_arc}, см. рис. 2
							   \begin{equation}
							   		\label{equation:penalty_d} 
											\varphi(M) = 
												\begin{cases}
													\displaystyle \frac{C}{M_1} \cdot M, & 0\leq M \leq M_1; \\
													\displaystyle \frac{1-C}{M_2-M_1}\cdot M-\frac{(1-C)M_1}{M_2-M_1},  & M_1\leq M \leq M_2 ; \\
													\displaystyle  1 , &M_2<M\leq 1.
												\end{cases}									
								\end{equation}		
			
		
			\begin{center}
					\begin{tikzpicture}
						\label{tikzpicture:gr1}
							\begin{axis}[
													title = $\varphi(M)$,
													xlabel = {$M$},
													ylabel = {$\varphi$},
													%%minor tick num = 1,
													%%domain=0:1,
													%%xmin=0, xmax=1.1, ymin=0, ymax=1.1,
													mark=*,
													grid=major,
													symbolic x coords={0,,$M_1$,,$M_2$,1},
													symbolic y coords={0,$C$,,,1}
												]
								%%\addplot[red] {(atan(x-0.01))/(atan(1-0.01)) + (atan(0.01))/(atan(1-0.01))};
									\addplot+ [red, mark=*] coordinates {(0,0) ($M_1$,$C$) ($M_2$,1)(1,1)};
									\draw[blue,dashed,thick] (axis cs:$M_2$,0) rectangle (axis cs:1,1);
									\draw[blue,dashed,thick] (axis cs:0,0) rectangle (axis cs:$M_1$,$C$);
			%%						\draw[blue,dashed,thick] (axis cs:0,1) line (axis cs:0.7,1);
									%%  \addplot coordinates {(0.5,0.2)};				
									\end{axis}
					\end{tikzpicture}
									\captionof{figure}{График дискретной функции $\varphi$}
			\end{center}	
		
		
			Функция (\ref{equation:penalty_d}) позволяет легко ``обнулять'' надбавки следующими после текущего мастер-узлам за счёт уменьшения $M_2$, и при необходимости усиливать штрафование узлов с малым $M$ за счёт уменьшения $C$ и, возможно, увеличения $M$.
			
			Благодаря функции (\ref{equation:penalty_d}) рост надбавки в случае закрытия $\hat{m}$, блоков $\hat{m}$ мастер-узлами   узлами  из $B_{\widetilde{n}}$ подряд может быть полностью остановлен.
			
			Усилить этот эффект можно изменениями привали мотивации: накопленную надбавку получает на первый мастер мастер-узел из $B_{{n}} \backslash B_{\widetilde{n}}$, выбранный после мастер-узла из $B_{\widetilde{n}}$, а первый мастер-узел, для которого $\varphi = 1$. В этом случае получение каким-либо мастер-узлом чрезмерной надбавки на $\forall$ конечном интервале времени становится практически невероятным. 
			
			Рассмотрим множество $B_{\hat{n}}$, состоящее из узлов, которые также могут отключаться, как и узлы из $B_{\widetilde{n}}$, но не преднамеренно, а по техническим причинам и стремящихся эт неполадки устранить. Узлы, не стремящиеся устранить неполадки, относятся к множеству $B_{\bar{n}}  \subset B_{\widetilde{n}}$.
			
			Допустим, что состав $B_{\hat{n}}$ полностью обновляется за среднее время $\Delta T$. К обновляемым относится также множество $B_{\hat{n}}=\varnothing$. Примерно за время 
				   \begin{equation}
							\label{equation:social} 
							\tau=\left(
										\left[
											\frac{n-\bar{n}}{\hat{n}}
										\right]+1
									  \right)
							\Delta T									
			\end{equation}		
			Все узлы из $B_{{n}} \backslash B_{\widetilde{n}}$ пройдут по одному разу устранение неполадок, при условии, что время работы $\forall$ узла без неполадок $\geq \tau$. Через время $l \cdot \tau$, где $l$ - достаточно большое натуральное число, все указанные узлы с равной вероятностью$l$ раз подвергнутся воздействию мотивировочного правила, которое будет дополнительным стимулом для как можно более оперативного устранения неполадок. 
			
			Следовательно, все узлы из $B_{{n}} \backslash B_{\widetilde{n}}$ находятся в равных условиях с точки зрения применения мотивировочного правила и с учётом результатов, полученных выше, можно сделать вывод о том, что распределение вознаграждения между узлами будет соответствовать их вкладу в работу сети.
					
				
										
\begin{thebibliography}{1}
	\bibitem{key-1} Satoshi Nakamoto (2009). ``Bitcoin: A Peer-to-Peer
	Electronic Cash System''. www.bitcoin.org .
	

	. 
\end{thebibliography}
\end{document}
