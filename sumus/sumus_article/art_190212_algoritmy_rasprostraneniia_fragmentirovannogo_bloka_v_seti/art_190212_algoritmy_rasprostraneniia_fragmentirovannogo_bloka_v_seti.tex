\documentclass{article}

\usepackage[english, russian]{babel} 
\usepackage[utf8]{inputenc} % Включаем поддержку UTF8

\usepackage[margin=2cm]{geometry}
\usepackage{indentfirst} % Красная строка после заголовка

\usepackage{graphicx,xcolor} % графика для svg
\graphicspath{{fig/}}

\title{Алгоритмы распространения блока в сети.}
\author{engi@sumus.team, a@sumus.team}
\date{12 февраля 2019 г.}

\begin{document}

\maketitle

\begin{abstract}
В статье рассматриваются алгоритмы распространения блока информации по сети с фиксированной топологией и ограниченной пропускной способностью. 
\end{abstract}

\section{Обозначения}

Введём обозначения:
\begin{itemize}
  \item $N$ --- общее количество узлов
  \item $M$ --- количество ``нисходящих'' связей на один узел
  \item $S$ --- количество уровней, первичный узел не учитывается
  \item $T$ --- время передачи одного (целого) блока
  \item $K$ --- количество фрагментов блока
  \item $R$ --- время распространения блока в сети
\end{itemize}

Формулы, связывающие $N$, $M$, $S$. Точные:
\begin{equation} N = \frac{M^S-1}{M-1} \label{equation:N_of_MS_exact} \end{equation}
\begin{equation} S = \frac{\log(N(M-1)+1)}{\log M} \label{equation:S_of_NM_exact} \end{equation}
Упрощённые:
\begin{equation} N \approx M^{S-1} \label{equation:N_of_MS} \end{equation}
\begin{equation} S \approx \frac{\log N}{\log M}+1 \label{equation:S_of_MS} \end{equation}

\begin{figure}[h]
	\centering
	\def\svgwidth{15cm} % изменить размер
	\input{fig/netw_m1.pdf_tex}
	\caption{Модель сети.}
	\label{figure:network_model}
\end{figure}

\section{Описания алгоритмов}

Общие условия распространения (фрагмента) блока в сети:
\begin{itemize}
	\item от одного узла нельзя одновременно передавать более одного блока
	\item один узел не может одновременно принимать более одного блока
	\item узлы могут передавать блоки только по заданным связям
	\item узел может начать передачу блока только после полного приёма этого блока от другого узла
	\item узел \textbf{0} содержит блок в начальный момент времени
\end{itemize}

\subsection{Алгоритм 00 (\textbf{A00})}

\begin{itemize}
	\item каждый узел (кроме узлов последнего уровня) передаёт блок ``своим'' узлам следующего уровня в порядке нумерации
\end{itemize}

\subsection{Алгоритм 01 (\textbf{A01})}

\begin{itemize}
	\item каждый узел передаёт блок ``своим'' узлам следующего уровня
	\item количество узлов следующего уровня неограниченно
\end{itemize}

\subsection{Алгоритм 10 (\textbf{A10})}

\begin{itemize}
	\item блок разделён на фрагменты одинакового размера
	\item каждый узел (кроме узлов последнего уровня) передаёт фрагменты блока в порядке перечисления: узел, фрагмент
\end{itemize}

\subsection{Алгоритм 11 (\textbf{A11})}

\begin{itemize}
	\item блок разделён на фрагменты одинакового размера
	\item каждый узел (кроме узлов последнего уровня) передаёт фрагменты блока в порядке перечисления: узел, фрагмент
\end{itemize}

\section{Расчёт}
При заданной модели сети, узел получающий блок последним --- узел последнего уровня, последний по порядку. Определив время получения блока этим узлом, получим время распространения блока в сети $R$.

\subsection{Алгоритм 00 (\textbf{A00})}

Время получения блока первым узлом первого уровня $T$. Время получения блока вторым узлом первого уровня $2T$. Время получения блока последним узлом первого уровня $R(1) = MT$.

Каждый следующий уровень увеличивает время $R$ распространения блока в сети на величину $MT$.

Время получения блока последним узлом уровня $s$ будет $R(s) = MTs$.

Время получения блока последним узлом последнего уровня $S$ будет $R(S) = MTS$.

\begin{equation} R_{A00} = MTS \label{equation:RA00} \end{equation}

\subsection{Алгоритм 10 (\textbf{A10})}

Главным отличием алгоритма \textbf{A10} от \textbf{A00} является возможность передавать фрагменты блока, не имея блок целиком.

Время передачи блока на первый уровень не отличается от времени по алгоритму \textbf{A00} и составляет $R(1) = MT$.

Рассмотрим ретрансляционный узел подробно. После получения фрагмента, узел имеет достаточно времени, чтобы передать этот фрагмент всем ``своим'' узлам следующего уровня до получения следующего фрагмента. Передача фрагмента на следующий уровень занимает $MT\frac{1}{K}$ времени. Каждый следующий уровень увеличивает время $R$ распространения блока в сети на эту величину $MT\frac{1}{K}$.

Время получения блока последним узлом уровня $s$ будет $R(s) = MT(\frac{s-1}{K} + 1)$.

Время получения блока последним узлом последнего уровня $S$ будет $R(S) = MT(\frac{S-1}{K}+1)$.

\begin{equation} R_{A01} = MT(\frac{S-1}{K}+1) \label{equation:RA01} \end{equation}

Если задать тривиальное значение $K = 1$, то формула получит вид (\ref{equation:RA00}).

Если величина $K \gg 1$ то распространение блока в сети может существенно приблизиться к времени передачи блока между соседними узлами.

\section{Визуализация}

\section{Реализация}

\subsection{Алгоритм 10 (\textbf{A10})}

Реализация алгоритма \textbf{A10} \dots

Блок разделяется на фрагменты $f_i$, из каждого фрагмента вычисляем хеш $h_i$. Полученные хеши подписываются узлами консенсуса. Таким образом получается содержательная часть \textbf{анонса} процесса, который распространяется по сети. Потом производится распространение фрагментов. 

Узел вначале получает \textbf{анонс} процесса, проверяет его подписи и использует для получения фрагментов блока. Получив фрагмент блока $f_i$, узел вычисляет $h_i$ и сверяет с хешем из \textbf{анонса}. При получении фрагмента нет проверки подписи. Проверка подписи происходит только при получении анонса, и у большинства узлов есть дополнительное время расчёта подписи до приёма первого фрагмента блока.

\end{document}
