\documentclass[a4paper, 12pt]{article}
\usepackage[utf8]{inputenc}
\usepackage[T2A]{fontenc}
\usepackage[english, russian]{babel} 
\usepackage{amscd, amsmath, amssymb, amsthm, enumerate, indentfirst, longtable, ifthen} %% addlibrary rom mathematic 
\usepackage{ragged2e}
\usepackage{microtype}
%кодировка шрифта в pdf
\usepackage{cmap}
%\usepackage{fontspec}
%\setmainfont[Ligatures=TeX]{Georgia}
%\setsansfont[Ligatures=TeX]{Arial}
\usepackage[most]{tcolorbox}
\usepackage{hyperref} %hyperlink support
\definecolor{black}{rgb}{0.0, 0.0, 0.0} 
\definecolor{lightyellow}{rgb}{1.0, 1.0, 0.87}
%\pagecolor{black}
%\color{lightyellow}
\pagecolor{lightyellow}
\color{black}
\usepackage{indentfirst}
\usepackage{setspace}
\setstretch{1.15} % interval 
 % \usepackage{multirow}
 % \usepackage{longtable} 
\RequirePackage{caption}
\DeclareCaptionLabelSeparator{defffis}{. }
\captionsetup{justification=centering,labelsep=defffis}
\usepackage{graphicx}
\graphicspath{{pictures/}}
\DeclareGraphicsExtensions{.png,}
 %%end picture 
%\linespread{1.3} % полуторный интервал
\usepackage{geometry}
\geometry{
 	tmargin=15mm,       % Up 
 	bmargin=15mm,      % Down 
 	lmargin=15mm,        % left поле
 	rmargin=15mm,        % right  
}
%%управление интервалами
\makeatletter
\renewcommand{\@listI}{%
	\topsep=0pt }
\makeatother
\makeatletter
\let\old@itemize=\itemize
\def\itemize{\old@itemize
	\setlength{\itemsep}{0pt}
	\setlength{\parskip}{0pt}
	\setlength{\leftskip}{10pt}
}
\makeatother
\makeatletter
\let\old@enumerate=\enumerate
\def\enumerate{\old@enumerate
	\setlength{\itemsep}{00pt}
	\setlength{\parskip}{0pt}
	\setlength{\leftskip}{0pt}
}\makeatother

%гиперссылки
\usepackage{xcolor}
\usepackage{hyperref}
\definecolor{linkcolor}{HTML}{799B03} % цвет ссылок
\definecolor{urlcolor}{HTML}{799B03} % цвет гиперссылок
\hypersetup{pdfstartview=FitH,  linkcolor=linkcolor,urlcolor=urlcolor, colorlinks=true}

%% new command 
\newcommand{\jj}{\righthyphenmin=20 \justifying}
%\renewcommand{\rmdefault}{ftm} % Times New Roman
%\renewcommand{\familydefault}{\sffamily}
%\renewcommand{\rmdefault}{cmr}
%\renewcommand{\rmdefault}{cmr}
%\renewcommand{\sfdefault}{cmss}
\renewcommand{\ttdefault}{cmtt}
\renewcommand{\rmdefault}{cmss}
\renewcommand{\ttdefault}{cmss}

\usepackage{float}
\usepackage{graphicx,xcolor} % графика для svg
\graphicspath{{fig/}}

%opening
\title{\Huge\textbf{AUREUS}\\
{Fast, free transaction and secure coin}}
\date{1 декабря 2020 г.}

\begin{document}

\maketitle

\textbf{\Large{Tech Paper v.0.2}}

\author{Aureus Securities AG}

{Short version: www.swissaureus.com}

\section{Введение}\addcontentsline{toc}{section}{Введение}

Почему и зачем мы создаётся блокчейн Aureus Securities AG? Существующие блокчейны при всей их привлекательности в части безопасности, анонимности имеют множество недостатков для практического применения:
\begin{itemize}
	\item Высокие комиссии за проведение транзакций.
	\item Полулегальные способы покупки/обмена токенов.
	\item Высокую волатильность.
	\item "Cложные" эмиссионные алгоритмы, постоянно влияющие на конечную стоимость токенов.  
	\item Не прогнозируемое время включения транзакции в блокчейн.
\end{itemize}

Блокчейн Aureus Securities AG создавался с целью сохранения важных свойств блокчейна и устранения его слабых сторон.

\textbf{Поэтому мы создали пространство со следующим набором свойств}

\begin{itemize} 
	\item Средства расчёта со стабильной ценой, привязанные к существующим денежным единицам.
	\item Прозрачная эмиссия.
	\item Бесплатные транзакции.
	\item Быстрые транзакции, которые гарантированно попадают в блокчейн в течение одной минуты.
	\item Децентрализация, основанная большом количестве узлов размещённым по всему миру.
\end{itemize}

Этот инструмент позволит людям:

\begin{itemize}
	\item быть уверенным в сохранности своих ценностей;
	\item свободно, быстро, надёжно и без комиссий переносить ценность в любую точку мира.

\end{itemize}

\section{Описание платформы}
Финансовая платформа Aureus Securities AG создана на технологиях блокчейн-конструктора 
Gaus\footnote{Описание всех возможностей блокчейн-конструктора Gaus не является целью данного документа.}.
Проектно-зависимая часть блокчейна была разработана совместно со специалистами Gaus.
Основные возможности блокчейн-сети Aureus Securities AG:
\begin{itemize}
	\item Используется алгоритм консенсуса sdBFT\footnote{Алгоритм семейства POS блокчейнов};
	\item Количество узлов участвующих в формировании блока - 5;
	\item Блок в блокчейне формируется каждые 20 секунд;
	\item Количество узлов в сети на начало старта 50;
	\item Выполнение привилегированных операций в сети осуществляется через multisign кошельки доверенными представителями компании  Aureus Securities AG;
	\item Привилегированными операциями используются для добавление/исключение узлов из блокчейн-сети, регистрация новых токенов, дополнительной эмиссии токенов;
	\item Multisign кошельки для пользователей блокчейна;
	\item Автоматическое усечение блокчейна (по достижению размера блокчейна 4х Tb );
	\item Скоростные характеристики блокчейна: включение в блок транзакций ~ до 10 000 в секунду (при обеспечении надлежащих каналов вязи и достаточной производительности узлов блокчейн сети).
\end{itemize}


\section{Алгоритм эмиссии}

В платформе Aureus Securities AG создаются токены, которые используются для расчётов. Основной токен системы AUS равен по стоимости 1 USD.
Эмиссия токенов осуществляется специальной multisign транзакцией, которая должна быть последовательно подписана доверенными представителями Aureus Securities AG  (кошельками имеющими специальный тэг Owner). Для того, чтобы блокчейн осуществил эмиссию должна быть выпущена и подписана транзакция \textit{multi sign register token transaction} как минимум тремя доверенными представителями. Доверенными представителями Aureus Securities AG являются штатные сотрудники различных подразделений, которые по своей совокупности не могут вступить в сговор и выполнить несанкционированную операцию. Также в структуре Aureus Securities AG  предусмотрен внешний контроль за операциями эмиссии со стороны службы безопасности. 

Характеристики эмиссии:
\begin{itemize}
	\item Создаваемый токен имеет трёх буквенное обозначение и десятичный код. 
	\item Каждая дополнительная эмиссия эмитируется по прозрачному алгоритму.
	\item Количество разрядов целой части: 11.
	\item Количество разрядов после запятой: 2.
\end{itemize}

При запуске блокчейн-сети, в генезис блоке блокчейна не определены токены и их количество. Первичная эмиссия - это создание токена AUS в количестве 1. Также будут выпущены токены EUR/USD/CHF для осуществления внутренних расчётов. 

В рамках финансового аудита Aureus Securities AG будет публиковаться информация о соответствии проведённых эмиссий токенов зарезервированным финансовым ресурсам.

\section{Бесплатные быстрые транзакции}

Транзакции в сети Aureus Securities AG бесплатны, блок закрывается за 20 секунд,  пропускная способность - до 10 000 транзакций в секунду. Оптимизация блокчейна происходит за счет того, что в блокчейне не хранится информация о входе-выходе транзакции (в отличие от UTXO-модели в BTC-подобных системах).

Бесплатность транзакций для пользователей возможна благодаря механизму минтинга, который реализован на алгоритме sdBFT.

\subsection{Reflection point}
\begin{itemize}
	\item Благодаря нулевой комиссии за пересылку токенов, система позволяет совершать микротранзакции, поэтому для обеспечения длительного функционирования блокчейна вводится процедура его периодического усечения.
	\item При достижении размера блокчейна в 4 Tb включается алгоритм усечения: система условно удаляет из рабочей части блокчейна  все нулевые (пустые) кошельки и все кошельки, имеющие баланс менее 1 токена. Состояние блокчейна после очистки будет записано в новый Genesis Block. Предыдущее состояние блокчейна будет сохранено только на специализированных (архивных) узлах.
\end{itemize}



\section{Блокчейн}

\subsection{Обоснование выбора}
Цели применения блокчейна Aureus Securities AG потребовали консенсуса, отвечающего следующим параметрам.
\begin{enumerate}
			\item Время создания нового блока не менее 30 секунд.
			\item Общее количество узлов, которые могут принять участие в выработке консенсуса должно быть не менее $10^3$.
			\item Высокая скорость транзакций – не менее $10^3$ транзакций в секунду.
			\item Реализация алгоритма блокчейна не должна требовать существенных вычислительных, по сравнению с блокчейнами PoW, мощностей.
\end{enumerate}
Выбор пал на алгоритм sdBFT, который  обладает более высоким быстродействием по сравнению с другими BFT алгоритмами. 
Потенциально большое число участников консенсуса усложняет предварительный сговор, когда группа голосующих узлов формирует новый блок, управляя составом блока по своему усмотрению, так как при следующем установлении консенсуса будет выбрано другое множество голосующих узлов. Псевдослучайный выбор множества голосующих узлов не позволит оказать существенного влияния на выбор узлов при следующем голосовании.
С описанием алгоритма можно ознакомиться в статье \href{magnet:?xt=urn:btih:c5a3d2762bd1f10c18f51b2606b1a32549d79ed4\&dn=Article\%20concensus\%20sdBFT.pdf\&tr=udp\%3A\%2F\%2Ftracker.leechers-paradise.org\%3A6969\&tr=udp\%3A\%2F\%2Ftracker.coppersurfer.tk\%3A6969}{Article concensus sdBFT}.

\subsection{Краткое описание алгоритм формирования нового блока}

\begin{itemize}
	\item Пусть в некий момент времени пользователь формирует транзакцию $I$. 
	\item Транзакция передается ближайшей ноде, с которой связан данный клиент. 
	\item Нода может находиться в одном из трёх состояний: пассивная, эскорт или мастер. 
	\item Если нода пассивная, она проверяет транзакцию и передает ее далее по пиринговой сети, пока транзакция не дойдёт до эскорт-ноды. 
	\item Эскорт-нода пересылает транзакцию мастер-ноде. 
	\item Мастер-нода проверяет транзакцию и, если транзакция корректная, пересылает ее эскорт-нодам, а также записывает транзакцию $I$ в формируемый блок. 
	\item Эскорт-ноды, приняв транзакцию $I$, проверяют ее на корректность и записывают ее в формируемый блок. 
	\item Данная последовательность действий повторяется до момента завершения блока, не более 20 секунд. 
	\item После этого мастер-нода рассылает сообщение о завершении блока. 
	\item Каждая эскорт-нода рассчитывает хеш блока транзакций, электронную подпись хеша и пересылает полученный хеш мастер-ноде.
	\item Мастер-нода рассчитывает количество корректных, по ее мнению, электронных подписей. Если полученное число корректных подписей превышает 2/3 от общего значения эскорт-нод, участвующих в консенсусе,  блок считается сформированным. Иначе блок не формируется.
	\item Блокчейн находится вне времени, не проверяет и не согласовывает время транзакций, помещаемых в блок.
	\item Системы, работающие поверх блокчейна, будут ориентироваться на некое усредненное время закрытия блоков (около 20 секунд).
\end{itemize}
    Алгоритм консенсуса sdBFT в реализации намного сложнее, для более полного понимания необходимо обратится к разделу  \hyperref[ExtAlg_sdBFT]{"Пояснение алгоритма работы консенсуса"}.

\subsection{Криптография}

\subsubsection{Алгоритмы электронной подписи и хеширования}
Применяемые криптографические алгоритмы соответствуют самым высоким требованиям по защите конфиденциальной и банковской информации. 
\begin{itemize}
	\item  Edwards-Curve Digital Signature Algorithm (EdDSA) \footnote{RFC 8032 https://tools.ietf.org/html/rfc8032};
	\item SHA-3 \footnote{FIPS PUB 202 https://nvlpubs.nist.gov/nistpubs/FIPS/NIST.FIPS.202.pdf};
\end{itemize}

\subsubsection{Генератор псевдослучайных чисел }

Стандартные генераторы псевдослучайных чисел, встроенные в операционные системы, как правило, имеют ряд существенных уязвимостей, наиболее опасные из которых:

\begin{itemize}
	\item В качестве seed для создания псевдослучайного числа используется timestamp. В итоге, если злоумышленник знает алгоритм генерации псевдослучайного числа и примерное время его генерации, то он может с высокой вероятностью методом перебора подобрать приватный ключ (пароль), сгенерированный таким алгоритмом.
	\item Даже если помимо timestamp используются иные данные, стандартные генераторы псевдослучайных чисел генерируют довольно предсказуемые последовательности, что предоставляет злоумышленниками возможность подбора паролей (хешей) методом перебора.	
\end{itemize}
Используемый в блокчейне программный генератор случайных чисел, основан на двойном вычислении хеш функции с динамическим изменением начального состояния. В качестве начального состояния используются данные имеющие характеристику, как у случайных процессов.  Качество случайной последовательности, вырабатываемой генератором псевдослучайной последовательности, не хуже $0.5+D$ на бинарный знак при $|D| <0.01$, что удовлетворяет гипотезе о равномерном распределении анализируемой последовательности случайных чисел.

\subsubsection{Типы транзакций}

\begin{itemize}
	\item Отправка токенов Aureus Securities AG.
	\item Анонс на регистрацию ноды.
	\item Анонс об исключении ноды.
	\item Новый genesis block (усечение).
	\item Multisign транзакция на регистрацию нового токена.
	\item Multisign транзакция эмиссии/обратной эмиссии токена.
	\item Информационная транзакция, транзакция содержащая произвольное текстовое поле, например адрес размещения отчёта аудиторов. 
\end{itemize}

\subsubsection{Кошельки}

\begin{itemize}
	\item Адрес кошелька - последовательность символов в кодировке Base58checkerMod2, которая записывается в транзакции, размещаемой в блокчейне.
	\item На кошелек можно получать и принимать токены зарегистрированные в блокчейне Aureus Securities AG.
	\item Типы кошельков:
\begin{itemize}
		\item Web:\begin{itemize}
		\item Для расчетов (транзакций) и проверки баланса.
		\item Доступен на защищённом  ресурсе компании Aureus Securities AG по адресам \href{https://wallet.swissaureus.com}{https://wallet.swissaureus.com}, \href{https://wallet.swissaureus.ch}{https://wallet.swissaureus.ch}. 
		\item При работе с  данным кошельком, отправка приватного ключа на сервера компании не производится. 
		
	\end{itemize}
		\item Легкий:\begin{itemize}
			\item Для расчетов (транзакций) и проверки баланса.
			\item Использует специальный протокол, который позволяет получать необходимые блоки и проверять при этом только дерево Меркла, а не весь блокчейн.
			
		\end{itemize}
	
	\item Стандартный:\begin{itemize}
		\item Хранит весь блокчейн.
		\item Может быть зарегистрирован как нода.
		
	\end{itemize}

\item Multisig:\begin{itemize}
	\item	Виртуальный кошелек, транзакцию с которого система принимает только при наличии нескольких подписей.
\end{itemize}

	
\end{itemize}

	\end{itemize}


\section{Практическое применение}

\subsection{Свободные международные платежи}

Рынок криптовалюты во многом вырос благодаря спросу на свободные быстрые и недорогие международные платежи. Особенно это касается Китая, где хождение валют сильно ограничено государством. \\

По итогам анализа участников конференции Money 20/20 в Сингапуре, ни одна платежная система не позволяет проводить трансграничные сделки размером более 5 000 USD. Единственный инструмент для перевода более 5 000 - 10 000 USD между государствами — SWIFT. Но для осуществления перевода потребуется заполнить множество документов, дождаться проверки документов банком, обсудить перевод с менеджером валютного контроля и затем ждать несколько часов для совершения перевода. Стоимость SWIFT составляет порядка 1\%.\\

Aureus Securities AG позволяет бесплатно переводить любые суммы в любую точку мира 
за несколько секунд без общения с менеджерами банков.

Чтобы получить токены блокчейна Aureus Securities AG нужно обратиться в офис по адресу. 
Чтобы получить вместо токенов денежные средства, потребуется пройти KYC авторизацию и после этого денежные средства будут направлены  по предоставленным  банковским реквизитам. 

\subsection{Использование инфраструктуры Aureus Securities AG}

Блокчейн  Aureus Securities AG является открытым, любой энтузиаст или компания может установить программное обеспечение узла блокчейн-сети. 
Дальше используя API для доступа к собственному узла блокчейн-сети можно  проводить транзакции токенами зарегистрированными в Aureus Securities AG, минуя инфраструктуры компании Aureus Securities AG.
Для развёртывания программного обеспечения и описания API работы с узлом блокчейна нужно обратится на сайт Aureus Securities AG.

Тем самым любой желающий может развернуть собственную систему поверх блокчейна Aureus Securities AG и использовать ее в своих целях. 

\section{Юридическая оболочка}

Legal Aureus Securities AG размещен на официальном сайте компании. 

\section{Пояснение алгоритма работы консенсуса \textit{sdbft}}\label{ExtAlg_sdBFT}

\begin{enumerate}
	
	\item
	Перед началом работы мы имеем пиринговую сеть с узлами, имеющими собственный сетевой адрес и уникальный номер, который знают все участники сети. Например, у нас будет 13 участников сети, пронумеруем все узлы номерами с 1 до 13. Так же мы договоримся, что в консенсус будет входить 5 узлов. 
	\begin{figure}[H]
		\centering
		\def\svgwidth{8cm} % изменить размер
		\input{fig/f1.pdf_tex}
		\label{F1}
		%	\caption{}
	\end{figure}
	Синим цветом будут обозначатся узлы, работающие в блокчейне. Зелёным цветом будут обозначаться узлы, участвующие в консенсусе. Мастер-узел будет помечаться оранжевым цветом. Узлы, находящиеся в нештатном режиме работы будут обозначаться красным цветом.
	
	\item
	Начало работы консенсуса. Пусть все узлы пиринговой сети примут блок 1. В принятом блоке содержится информация, которая позволит функции $f$ (см приложение А) создать случайную последовательность. Пусть эта последовательность будет следующей --- №3,5,7,11,13. Пометим цветом узлы, имеющие №3,5,7,11,13.
	\begin{figure}[H]
		\centering
		\def\svgwidth{8cm} % изменить размер
		\input{fig/f2.pdf_tex}
		\label{F2}
		%	\caption{}
	\end{figure}
	
	\item
	Как показано на рисунке выше узел №3 мы пометили оранжевым цветом, чтобы показать, что он является мастер-узлом. С этого момента узлы №3,5,7,11,13 участвуют в консенсусе.
	
	\item
	Пусть узел №3 получил новую транзакцию от узла №2. Узел №3 проверяет, является ли транзакция корректной, если она признается корректной, то узел №3 пересылает ее узлам №5,7,11,13.
	
	\item
	По завершению времени, отведённого на закрытие блока узел №3 пересылает узлам №5,7,11 и 13 сообщение о закрытии блока.
	\begin{figure}[H]
		\centering
		\def\svgwidth{8cm} % изменить размер
		\input{fig/f3.pdf_tex}
		\label{F3}
		%	\caption{}
	\end{figure}
	
	\item
	Узлы №5,7,11 и 13 пересылают хеш дерева Меркла, принятых ими транзакций, и свои подписи под хешем узлу №3.
	\begin{figure}[H]
		\centering
		\def\svgwidth{8cm} % изменить размер
		\input{fig/f4.pdf_tex}
		\label{F4}
		%	\caption{}
	\end{figure}
	
	\item
	Узел №3 считает подписи, если подписи корректны и их число удовлетворяет решению задачи византийских генералов, их не менее 3, то блок считается сформированным. Узел №3 рассылает анонс нового блока всем узлам сети.
	\begin{figure}[H]
		\centering
		\def\svgwidth{8cm} % изменить размер
		\input{fig/f5.pdf_tex}
		\label{F5}
		%	\caption{}
	\end{figure}
	
	\item
	Узлы принимают блок №2. Возвращаемся на второй шаг алгоритма, пусть теперь функция $f$ (см приложение А) создаст случайную последовательность на основании принятых блоков, пусть будут номера 6,2,3,11,12.
	\begin{figure}[H]
		\centering
		\def\svgwidth{8cm} % изменить размер
		\input{fig/f6.pdf_tex}
		\label{F6}
		%	\caption{}
	\end{figure}
	
	Далее процесс повторяется в соответствии с п.п. 3-8 алгоритма.
\end{enumerate}

\subsubsection{Обработка ошибочных ситуаций алгоритмом \textit{sdbft}}

\paragraph{Мастер узел недоступен}
\begin{enumerate}
	\item
	Повторим п.п. алгоритма 1 и 2.
	
	\item
	Перед началом работы мы имеем пиринговую сеть с узлами, имеющими собственный сетевой адрес и уникальный номер, который знают все участники сети. Например, у нас будет 13 участников сети, пронумеруем все узлы номерами с 1 до 13. Так же мы договоримся, что в консенсус будет входить 5 узлов.
	\begin{figure}[H]
		\centering
		\def\svgwidth{8cm} % изменить размер
		\input{fig/f1.pdf_tex}
		\label{F7}
		%	\caption{}
	\end{figure}
	
	\item
	Начало работы консенсуса. Пусть все узлы пиринговой сети примут блок 1. В принятом блоке содержится информация, которая позволит функции $f$ (см приложение А) создать случайную последовательность. Пусть эта последовательность будет следующей --- №3,5,7,11,13. Узел №3 недоступен. Пометим цветом узлы сети.
	\begin{figure}[H]
		\centering
		\def\svgwidth{8cm} % изменить размер
		\input{fig/f8.pdf_tex}
		\label{F8}
		%	\caption{}
	\end{figure}
	
	
	\item
	Узлы эскорта не получат сообщения о закрытия блока, блокчейн сеть переходит на следующий раунд (см приложение А).
	
	\item
	Возвращаемся на второй шаг алгоритма, пусть теперь функция $f$ (см приложение А) создаст случайную последовательность на основании принятого блока и номера раунда, пусть будут номера 7,1,4,12,13. Сеть блокчейна будет выглядеть следующим образом.
	\begin{figure}[H]
		\centering
		\def\svgwidth{8cm} % изменить размер
		\input{fig/f9.pdf_tex}
		\label{F9}
		%	\caption{}
	\end{figure}
	
	\item
	Далее алгоритм будет исполняться штатным образом в соответствии с п.п. 3-8.
	
\end{enumerate}

\paragraph{Эскорт узел недоступен}
\begin{enumerate}
	\item
	Повторим п.п. алгоритма 1 и 2.
	
	\item
	Перед началом работы мы имеем пиринговую сеть с узлами, имеющими собственный сетевой адрес и уникальный номер, который знают все участники сети. Например, у нас будет 13 участников сети, пронумеруем все узлы номерами с 1 до 13. Так же мы договоримся, что в консенсус будет входить 5 узлов.
	\begin{figure}[H]
		\centering
		\def\svgwidth{8cm} % изменить размер
		\input{fig/f1.pdf_tex}
		\label{F10}
		%	\caption{}
	\end{figure}
	
	\item
	Начало работы консенсуса. Пусть все узлы пиринговой сети примут блок №1. В принятом блоке содержится информация, которая позволит функции $f$ (см. приложение А) создать случайную последовательность. Пусть эта последовательность будет следующей --- 3,5,7,11,13. Узел 13 недоступен. Пометим выбранные узлы цветом.
	\begin{figure}[H]
		\centering
		\def\svgwidth{8cm} % изменить размер
		\input{fig/f11.pdf_tex}
		\label{F11}
		%	\caption{}
	\end{figure}
	
	\item
	Узел эскорта №13 не получит сообщения о закрытия блока и не пошлёт свою подпись для закрытия блока. Если оставшиеся три узла пошлют корректные подписи транзакций формируемого блока, то блок будет сформирован. Узлы, участвующие в консенсусе будут перевыбраны.
	
	\item
	Если оставшиеся три узла пошлют не корректные подписи транзакций формируемого блока, то блок не будет сформирован. Блокчейн перейдёт на следующий раунд. Узлы, участвующие в консенсусе будут перевыбраны.
	
\end{enumerate}

\paragraph{Поступила некорректная транзакция}
\begin{enumerate}
	\item
	Пусть узел 3 получил новую транзакцию от узла №2. Узел №3 проверяет, транзакцию и признает ее некорректной.
	
	\item
	Если узел №3 признал транзакцию некорректной, то он ее отбрасывает, сообщение узлу №2 о некорректной транзакции не пересылается на узлы, входящие в консенсус.
	\begin{figure}[H]
		\centering
		\def\svgwidth{8cm} % изменить размер
		\input{fig/f12.pdf_tex}
		\label{F12}
		%	\caption{}
	\end{figure}
	
\end{enumerate}

\paragraph{Количество блоков в блокчейне разное на разных узлах} \mbox{} \\ 
Разное количество принятых блоков на разных узлах блокчейна может быть разным в случае, например, если сеть была сегментирована и не все узлы успели синхронизироваться. Повторим п.п. 1-5 алгоритма.

\begin{enumerate}
	\item
	Перед началом работы мы имеем пиринговую сеть с узлами, имеющими собственный сетевой адрес и уникальный номер, который знают все участники сети. Например, у нас будет 13 участников сети, пронумеруем все узлы номерами с 1 до 13. Так же мы договоримся, что в консенсус будет входить 5 узлов.	
	\begin{figure}[H]
		\centering
		\def\svgwidth{8cm} % изменить размер
		\input{fig/f1.pdf_tex}
		\label{F13}
		%	\caption{}
	\end{figure}
	
	\item
	Начало работы консенсуса, пусть все узлы пиринговой сети примут блок 1.В принятом блоке содержится информация, которая позволит функции $f$ (см приложение А) создать случайную последовательность. Пусть эта последовательность будет следующей --- 3,5,7,11,13, узлы 7 и 11 имеют отличное число принятых блоков от узлов 3,5,13. Пометим выбранные узлы цветом.
	\begin{figure}[H]
		\centering
		\def\svgwidth{8cm} % изменить размер
		\input{fig/f14.pdf_tex}
		\label{F14}
		%	\caption{}
	\end{figure}
	
	\item
	Как показано на рисунке узел №3 мы пометили оранжевым цветом, чтобы показать, что он является мастер-узлом. С этого момента узлы 3,5,13 участвуют в консенсусе.
	
	\item
	Пусть узел 3 получил новую транзакцию от узла №2. Узел №3 проверяет, является ли транзакция корректной, если она признается корректной, то узел №3 пересылает ее узлам 5,7,11,13. Узлы 7 и 11 отвергают транзакцию.
	\begin{figure}[H]
		\centering
		\def\svgwidth{8cm} % изменить размер
		\input{fig/f15.pdf_tex}
		\label{F15}
		%	\caption{}
	\end{figure}
	
	\item
	По завершению времени на закрытие блока узел №3 пересылает узлам №5,7,11,13 сообщение о закрытии блока, узлы 7 и 11 отвергают сообщение.
	\begin{figure}[H]
		\centering
		\def\svgwidth{8cm} % изменить размер
		\input{fig/f16.pdf_tex}
		\label{F16}
		%	\caption{}
	\end{figure}
	
	\item
	Узлы №5 и 13 пересылают хеш транзакций и свои подписи под хешем узлу №3.
	
	\item
	Узел №3 проверяет подписи узлов эскорта. Так как количество подписей узлов эскорта недостаточно для принятия блока, то блокчейн переходит на следующий раунд.
	\begin{figure}[H]
		\centering
		\def\svgwidth{8cm} % изменить размер
		\input{fig/f17.pdf_tex}
		\label{F17}
		%	\caption{}
	\end{figure}
	
\end{enumerate}

\paragraph{Узел отвергает новый блок блокчейна} \mbox{} \\ 
Узел сети может отвергнуть новый блок блокчейна. Причин, по которым узел отвергает блок может быть несколько, например, на узле в следствии программного или аппаратного сбоя возникла ошибка чтения из базы данных и балансы кошельков изменились. Повторим п.7.

\begin{enumerate}
	\item
	Узел №3 рассылает анонс нового блока всем узлам сети.
	\begin{figure}[H]
		\centering
		\def\svgwidth{8cm} % изменить размер
		\input{fig/f5.pdf_tex}
		\label{F18}
		%	\caption{}
	\end{figure}
	
	\item
	Пусть узел №1 отвергает блок.
	\begin{figure}[H]
		\centering
		\def\svgwidth{8cm} % изменить размер
		\input{fig/f19.pdf_tex}
		\label{F19}
		%	\caption{}
	\end{figure}
	
	\item
	Узел №1 пытается найти в сети блок с отличной от признанного им ошибочным блока хеш-суммой, если узел не может найти удовлетворяющий его блок, то узел начинает процедуру пересинхронизации блокчейна см. п. 3.5.
	
\end{enumerate}

\paragraph{Сеть отвергает новый блок блокчейна} \mbox{} \\ 
При создании нового блока теоретически может произойти сознательная попытка группы узлов навязать собственный, ошибочный, блок. Пусть узлы №3,5,7,11,13 пытаются навязать собственный, неправильный блок сети блокчейна Повторим п.7.

\begin{enumerate}
	\item
	Узел №3 рассылает анонс нового блока всем узлам сети.
	\begin{figure}[H]
		\centering
		\def\svgwidth{8cm} % изменить размер
		\input{fig/f20.pdf_tex}
		\label{F20}
		%	\caption{}
	\end{figure}
	
	\item
	Все узлы сети отвергают новый блок.
	\begin{figure}[H]
		\centering
		\def\svgwidth{8cm} % изменить размер
		\input{fig/f21.pdf_tex}
		\label{F21}
		%	\caption{}
	\end{figure}
	
	\item
	Блокчейн переходит на следующий раунд и далее пока не будет корректно сформирован следующий блок.
	
\end{enumerate}

\paragraph{В сети появилось два блока с идентичными номерами} \mbox{} \\ 
Два блока с идентичными номерами могут возникнуть в сети только, если будет сформировано два консенсуса из узлов с разными раундами, а это возможно если в сети возник глобальный сбой. Например, часть узлов находилась на серверах которые были одновременно перезагружены. В таком случае будет происходить следующее.

\begin{enumerate}
	\item
	Предположим, что сформировалось два набора узлов для создания консенсуса - 3,5,7,11,13 и 8,9,6,1,10. Узел №3 и №8 рассылают анонс нового блока всем узлам сети.
	\begin{figure}[H]
		\centering
		\def\svgwidth{8cm} % изменить размер
		\input{fig/f22.pdf_tex}
		\label{F22}
		%	\caption{}
	\end{figure}
	
	\item
	Узел при принятии нового блока проверяет входит ли раунд создания блока в доверительный интервал раундов или нет. Т.е. насколько сильно раунд нового блока отличается от собственного раунда узла. Если раунд признается узлом корректным, то блок принимается. Если признается не корректным, то блок отвергается. Далее, возможно два пути развития ситуации: 
	\subparagraph{a.} Раунды сформированных блоков находятся в доверительном интервале, в таком случае узлом будет принят блок, пришедший первым. Вероятность попадания в блокчейн блока будет зависеть от того какой блок был принят большинством узлов сети.
	\subparagraph{b.} Раунд одного из сформированных блоков не находится в доверительном интервале у большинства улов сети. Следовательно, большинством узлов будет принят блок с номером раунда из доверительного интервала.
	
\end{enumerate}

\end{document}
