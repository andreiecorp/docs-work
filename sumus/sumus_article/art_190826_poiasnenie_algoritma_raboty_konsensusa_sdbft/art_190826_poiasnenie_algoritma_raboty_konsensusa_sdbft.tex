\documentclass[12pt,twoside,a4paper]{article}

\usepackage[utf8]{inputenc} % Включаем поддержку UTF8
\usepackage[russian]{babel} % [english, russian]

\usepackage[margin=2cm]{geometry}
\usepackage{indentfirst} % Красная строка после заголовка

\usepackage{float}

\usepackage{graphicx,xcolor} % графика для svg
\graphicspath{{fig/}}

\title{Пояснение алгоритма работы консенсуса \textit{sdbft}.}
\author{rr@sumus.team, a@sumus.team}
\date{26 августа 2019 г.}

\begin{document}

\maketitle

\setcounter{section}{1}
\setcounter{subsection}{1}
\subsubsection{Пояснение алгоритма работы консенсуса \textit{sdbft}} \mbox{} \\ 
Данный раздел содержит упрощенное описание алгоритма консенсуса \textbf{\textit{sdbft}}.

\begin{enumerate}

\item
Перед началом работы мы имеем пиринговую сеть с узлами, имеющими собственный сетевой адрес и уникальный номер, который знают все участники сети. Например, у нас будет 13 участников сети, пронумеруем все узлы номерами с 1 до 13. Так же мы договоримся, что в консенсус будет входить 5 узлов. 
\begin{figure}[H]
	\centering
	\def\svgwidth{8cm} % изменить размер
	\input{fig/f1.pdf_tex}
	\label{F1}
%	\caption{}
\end{figure}
Синим цветом будут обозначатся узлы, работающие в блокчейне. Зелёным цветом будут обозначаться узлы, участвующие в консенсусе. Мастер-узел будет помечаться оранжевым цветом. Узлы, находящиеся в нештатном режиме работы будут обозначаться красным цветом.

\item
Начало работы консенсуса. Пусть все узлы пиринговой сети примут блок 1. В принятом блоке содержится информация, которая позволит функции $f$ (см приложение А) создать случайную последовательность. Пусть эта последовательность будет следующей --- №3,5,7,11,13. Пометим цветом узлы, имеющие №3,5,7,11,13.
\begin{figure}[H]
	\centering
	\def\svgwidth{8cm} % изменить размер
	\input{fig/f2.pdf_tex}
	\label{F2}
%	\caption{}
\end{figure}

\item
Как показано на рисунке выше узел №3 мы пометили оранжевым цветом, чтобы показать, что он является мастер-узлом. С этого момента узлы №3,5,7,11,13 участвуют в консенсусе.

\item
Пусть узел №3 получил новую транзакцию от узла №2. Узел №3 проверяет, является ли транзакция корректной, если она признается корректной, то узел №3 пересылает ее узлам №5,7,11,13.

\item
По завершению времени, отведённого на закрытие блока узел №3 пересылает узлам №5,7,11 и 13 сообщение о закрытии блока.
\begin{figure}[H]
	\centering
	\def\svgwidth{8cm} % изменить размер
	\input{fig/f3.pdf_tex}
	\label{F3}
%	\caption{}
\end{figure}

\item
Узлы №5,7,11 и 13 пересылают хеш дерева Меркла, принятых ими транзакций, и свои подписи под хешем узлу №3.
\begin{figure}[H]
	\centering
	\def\svgwidth{8cm} % изменить размер
	\input{fig/f4.pdf_tex}
	\label{F4}
%	\caption{}
\end{figure}

\item
Узел №3 считает подписи, если подписи корректны и их число удовлетворяет решению задачи византийских генералов, их не менее 3, то блок считается сформированным. Узел №3 рассылает анонс нового блока всем узлам сети.
\begin{figure}[H]
	\centering
	\def\svgwidth{8cm} % изменить размер
	\input{fig/f5.pdf_tex}
	\label{F5}
%	\caption{}
\end{figure}

\item
Узлы принимают блок №2. Возвращаемся на второй шаг алгоритма, пусть теперь функция $f$ (см приложение А) создаст случайную последовательность на основании принятых блоков, пусть будут номера 6,2,3,11,12.
\begin{figure}[H]
	\centering
	\def\svgwidth{8cm} % изменить размер
	\input{fig/f6.pdf_tex}
	\label{F6}
%	\caption{}
\end{figure}

Далее процесс повторяется в соответствии с п.п. 3-8 алгоритма.
\end{enumerate}

\subsubsection{Обработка ошибочных ситуаций алгоритмом \textit{sdbft}}

\paragraph{Мастер узел недоступен}
\begin{enumerate}
\item
Повторим п.п. алгоритма 1 и 2.

\item
Перед началом работы мы имеем пиринговую сеть с узлами, имеющими собственный сетевой адрес и уникальный номер, который знают все участники сети. Например, у нас будет 13 участников сети, пронумеруем все узлы номерами с 1 до 13. Так же мы договоримся, что в консенсус будет входить 5 узлов.
\begin{figure}[H]
	\centering
	\def\svgwidth{8cm} % изменить размер
	\input{fig/f1.pdf_tex}
	\label{F7}
%	\caption{}
\end{figure}

\item
Начало работы консенсуса. Пусть все узлы пиринговой сети примут блок 1. В принятом блоке содержится информация, которая позволит функции $f$ (см приложение А) создать случайную последовательность. Пусть эта последовательность будет следующей --- №3,5,7,11,13. Узел №3 недоступен. Пометим цветом узлы сети.
\begin{figure}[H]
	\centering
	\def\svgwidth{8cm} % изменить размер
	\input{fig/f8.pdf_tex}
	\label{F8}
%	\caption{}
\end{figure}


\item
Узлы эскорта не получат сообщения о закрытия блока, блокчейн сеть переходит на следующий раунд (см приложение А).

\item
Возвращаемся на второй шаг алгоритма, пусть теперь функция $f$ (см приложение А) создаст случайную последовательность на основании принятого блока и номера раунда, пусть будут номера 7,1,4,12,13. Сеть блокчейна будет выглядеть следующим образом.
\begin{figure}[H]
	\centering
	\def\svgwidth{8cm} % изменить размер
	\input{fig/f9.pdf_tex}
	\label{F9}
%	\caption{}
\end{figure}

\item
Далее алгоритм будет исполняться штатным образом в соответствии с п.п. 3-8.

\end{enumerate}

\paragraph{Эскорт узел недоступен}
\begin{enumerate}
\item
Повторим п.п. алгоритма 1 и 2.

\item
Перед началом работы мы имеем пиринговую сеть с узлами, имеющими собственный сетевой адрес и уникальный номер, который знают все участники сети. Например, у нас будет 13 участников сети, пронумеруем все узлы номерами с 1 до 13. Так же мы договоримся, что в консенсус будет входить 5 узлов.
\begin{figure}[H]
	\centering
	\def\svgwidth{8cm} % изменить размер
	\input{fig/f1.pdf_tex}
	\label{F10}
%	\caption{}
\end{figure}

\item
Начало работы консенсуса. Пусть все узлы пиринговой сети примут блок №1. В принятом блоке содержится информация, которая позволит функции $f$ (см. приложение А) создать случайную последовательность. Пусть эта последовательность будет следующей --- 3,5,7,11,13. Узел 13 недоступен. Пометим выбранные узлы цветом.
\begin{figure}[H]
	\centering
	\def\svgwidth{8cm} % изменить размер
	\input{fig/f11.pdf_tex}
	\label{F11}
%	\caption{}
\end{figure}

\item
Узел эскорта №13 не получит сообщения о закрытия блока и не пошлёт свою подпись для закрытия блока. Если оставшиеся три узла пошлют корректные подписи транзакций формируемого блока, то блок будет сформирован. Узлы, участвующие в консенсусе будут перевыбраны.

\item
Если оставшиеся три узла пошлют не корректные подписи транзакций формируемого блока, то блок не будет сформирован. Блокчейн перейдёт на следующий раунд. Узлы, участвующие в консенсусе будут перевыбраны.

\end{enumerate}

\paragraph{Поступила некорректная транзакция}
\begin{enumerate}
\item
Пусть узел 3 получил новую транзакцию от узла №2. Узел №3 проверяет, транзакцию и признает ее некорректной.

\item
Если узел №3 признал транзакцию некорректной, то он ее отбрасывает, сообщение узлу №2 о некорректной транзакции не пересылается на узлы, входящие в консенсус.
\begin{figure}[H]
	\centering
	\def\svgwidth{8cm} % изменить размер
	\input{fig/f12.pdf_tex}
	\label{F12}
%	\caption{}
\end{figure}

\end{enumerate}

\paragraph{Количество блоков в блокчейне разное на разных узлах} \mbox{} \\ 
Разное количество принятых блоков на разных узлах блокчейна может быть разным в случае, например, если сеть была сегментирована и не все узлы успели синхронизироваться. Повторим п.п. 1-5 алгоритма.

\begin{enumerate}
\item
Перед началом работы мы имеем пиринговую сеть с узлами, имеющими собственный сетевой адрес и уникальный номер, который знают все участники сети. Например, у нас будет 13 участников сети, пронумеруем все узлы номерами с 1 до 13. Так же мы договоримся, что в консенсус будет входить 5 узлов.	
\begin{figure}[H]
	\centering
	\def\svgwidth{8cm} % изменить размер
	\input{fig/f1.pdf_tex}
	\label{F13}
%	\caption{}
\end{figure}
	
\item
Начало работы консенсуса, пусть все узлы пиринговой сети примут блок 1.В принятом блоке содержится информация, которая позволит функции $f$ (см приложение А) создать случайную последовательность. Пусть эта последовательность будет следующей --- 3,5,7,11,13, узлы 7 и 11 имеют отличное число принятых блоков от узлов 3,5,13. Пометим выбранные узлы цветом.
\begin{figure}[H]
	\centering
	\def\svgwidth{8cm} % изменить размер
	\input{fig/f14.pdf_tex}
	\label{F14}
%	\caption{}
\end{figure}

\item
Как показано на рисунке узел №3 мы пометили оранжевым цветом, чтобы показать, что он является мастер-узлом. С этого момента узлы 3,5,13 участвуют в консенсусе.

\item
Пусть узел 3 получил новую транзакцию от узла №2. Узел №3 проверяет, является ли транзакция корректной, если она признается корректной, то узел №3 пересылает ее узлам 5,7,11,13. Узлы 7 и 11 отвергают транзакцию.
\begin{figure}[H]
	\centering
	\def\svgwidth{8cm} % изменить размер
	\input{fig/f15.pdf_tex}
	\label{F15}
%	\caption{}
\end{figure}

\item
По завершению времени на закрытие блока узел №3 пересылает узлам №5,7,11,13 сообщение о закрытии блока, узлы 7 и 11 отвергают сообщение.
\begin{figure}[H]
	\centering
	\def\svgwidth{8cm} % изменить размер
	\input{fig/f16.pdf_tex}
	\label{F16}
%	\caption{}
\end{figure}

\item
Узлы №5 и 13 пересылают хеш транзакций и свои подписи под хешем узлу №3.
	
\item
Узел №3 проверяет подписи узлов эскорта. Так как количество подписей узлов эскорта недостаточно для принятия блока, то блокчейн переходит на следующий раунд.
\begin{figure}[H]
	\centering
	\def\svgwidth{8cm} % изменить размер
	\input{fig/f17.pdf_tex}
	\label{F17}
%	\caption{}
\end{figure}

\end{enumerate}

\paragraph{Узел отвергает новый блок блокчейна} \mbox{} \\ 
Узел сети может отвергнуть новый блок блокчейна. Причин, по которым узел отвергает блок может быть несколько, например, на узле в следствии программного или аппаратного сбоя возникла ошибка чтения из базы данных и балансы кошельков изменились. Повторим п.7.

\begin{enumerate}
\item
Узел №3 рассылает анонс нового блока всем узлам сети.
\begin{figure}[H]
	\centering
	\def\svgwidth{8cm} % изменить размер
	\input{fig/f5.pdf_tex}
	\label{F18}
%	\caption{}
\end{figure}

\item
Пусть узел №1 отвергает блок.
\begin{figure}[H]
	\centering
	\def\svgwidth{8cm} % изменить размер
	\input{fig/f19.pdf_tex}
	\label{F19}
%	\caption{}
\end{figure}

\item
Узел №1 пытается найти в сети блок с отличной от признанного им ошибочным блока хеш-суммой, если узел не может найти удовлетворяющий его блок, то узел начинает процедуру пересинхронизации блокчейна см. п. 3.5.

\end{enumerate}

\paragraph{Сеть отвергает новый блок блокчейна} \mbox{} \\ 
При создании нового блока теоретически может произойти сознательная попытка группы узлов навязать собственный, ошибочный, блок. Пусть узлы №3,5,7,11,13 пытаются навязать собственный, неправильный блок сети блокчейна Повторим п.7.

\begin{enumerate}
\item
Узел №3 рассылает анонс нового блока всем узлам сети.
\begin{figure}[H]
	\centering
	\def\svgwidth{8cm} % изменить размер
	\input{fig/f20.pdf_tex}
	\label{F20}
%	\caption{}
\end{figure}

\item
Все узлы сети отвергают новый блок.
\begin{figure}[H]
	\centering
	\def\svgwidth{8cm} % изменить размер
	\input{fig/f21.pdf_tex}
	\label{F21}
%	\caption{}
\end{figure}

\item
Блокчейн переходит на следующий раунд и далее пока не будет корректно сформирован следующий блок.

\end{enumerate}

\paragraph{В сети появилось два блока с идентичными номерами} \mbox{} \\ 
Два блока с идентичными номерами могут возникнуть в сети только, если будет сформировано два консенсуса из узлов с разными раундами, а это возможно если в сети возник глобальный сбой. Например, часть узлов находилась на серверах которые были одновременно перезагружены. В таком случае будет происходить следующее.

\begin{enumerate}
\item
Предположим, что сформировалось два набора узлов для создания консенсуса - 3,5,7,11,13 и 8,9,6,1,10. Узел №3 и №8 рассылают анонс нового блока всем узлам сети.
\begin{figure}[H]
	\centering
	\def\svgwidth{8cm} % изменить размер
	\input{fig/f22.pdf_tex}
	\label{F22}
%	\caption{}
\end{figure}

\item
Узел при принятии нового блока проверяет входит ли раунд создания блока в доверительный интервал раундов или нет. Т.е. насколько сильно раунд нового блока отличается от собственного раунда узла. Если раунд признается узлом корректным, то блок принимается. Если признается не корректным, то блок отвергается. Далее, возможно два пути развития ситуации: 
\subparagraph{a.} Раунды сформированных блоков находятся в доверительном интервале, в таком случае узлом будет принят блок, пришедший первым. Вероятность попадания в блокчейн блока будет зависеть от того какой блок был принят большинством узлов сети.
\subparagraph{b.} Раунд одного из сформированных блоков не находится в доверительном интервале у большинства улов сети. Следовательно, большинством узлов будет принят блок с номером раунда из доверительного интервала.

\end{enumerate}

\end{document}
