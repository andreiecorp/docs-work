\documentclass[a4paper, 12pt]{article}

\usepackage[english, russian]{babel} 
\usepackage[utf8]{inputenc} % Включаем поддержку UTF8

\usepackage[margin=2cm]{geometry}
\usepackage{indentfirst} % Красная строка после заголовка

\usepackage{amsfonts} % разные математические шрифты

\usepackage{graphicx,xcolor} % графика для svg
\graphicspath{{fig/}}

\title{Метод усечения блокчейна в пространстве состояний}
\author{noise@sumus.team}
\date{1 ноября 2018 г.}

\begin{document}
\maketitle
%
\begin{abstract}
% Правило мотивации постоянного подключения узла к сети \dots \dots \textit{добавить}
Предлагается системный подход к процессу усечения блокчейна, основанный на теории пространства состояний динамической системы. Рассматривается усечение блокчейна двух видов: по времени, что позволяет условно ``отбросить'' предысторию системы и по фазовым координатам, что позволяет уменьшить размерность пространства состояний на некотором интервале времени.
\end{abstract}

\section{Введение}\addcontentsline{toc}{section}{Введение}
При длительной истории (отдельно взятой) конкретной истории блокчейн и большом количестве узлов рано или поздно возникает проблема вычислительных ресурсов, связанная с необходимостью поддерживать в постоянной готовности к доступу всю информацию, накопленную за всю историю и обо всех сущностях системы, включая ``обнулившиеся'' кошельки, ``спящие'' узлы и тому подобное. Даже при высоких возможностях современных компьютерных систем это приводит к появлению некоторого ``балласта'', на поддержание информации о котором в постоянной готовности тратятся значительные ресурсы. Для решения данной проблемы предлагается процедура ``усечения'' блокчейна, позволяющая отбросить неактуальные фрагменты, но и сохранить ``прозрачность'' системы для пользователей и полную преемственность информации. Для достижения этой цели предлагается использовать описание блокчейна как системы в пространстве состояний.

\section{Основные положения и допущения}\addcontentsline{toc}{section}{Основные положения и допущения}
Пусть задано $n$-мерное линейное метрическое пространство $\mathbb{X}$ с метрикой $\rho(x_1,x_2)$, где $x_1,x_2 \in \mathbb{X}$. Пусть оно является пространством состояний системы $S$, состоящей их $r$ линейнозависимых элементов $S_l, \: l=1,\dots,r \: $, каждый из которых может быть представлен точкой подпространства $\mathbb{X}_l, \dim \mathbb{X}_l = k_l$ пространства $\mathbb{X}, \sum\limits _{l=1} ^{r} k_l=n$. В этом случае состояние системы $S$ в момент времени $t$ может быть представлено точкой $A(x_1,\dots,x_n)$ пространства $\mathbb{X}$, причём радиус-вектор $\vec{a}$ этой точки является блочным вектором $\vec{a}=(x_{11},\dots,x_{1k_1}|x_{21},\dots,x_{2k_2}|x_{r1},\dots,x_{rk_r})$, где каждый ``блок'' вектора соответствует  ``своему'' подпространству.
	
Траекторией системы $S$ будем называть кривую $\Gamma \subset \mathbb{X}$, начальная точка которой $A_0$ соответствует моменту времени $t_0$, конечная точка $A_1$ --- $t_1$. Таким образом, траектория $\Gamma$ соответствует отрезку $[t_0,t_1]$, а $t$ является параметром, откладываемым вдоль кривой.
	
\begin{figure}[!h]
	\centering
	\def\svgwidth{15cm} % изменить размер
	\input{fig/f1.pdf_tex}
	\caption{Траектория системы $S$ в пространстве $\mathbb{X}$.}
	\label{Traektoriia_S}
\end{figure}

На рис.~\ref{Traektoriia_S} изображена траектория системы $S$ при $n=3,\:r=2,\:k_1=2,\:k_2=1$. Значение координаты $x_{ij}(1 \le i \le r; 1 \le j \le k_l, l=1,\dots,r)$ будем называть ``$\varepsilon_{ij}$-нулевым'', если при заданном малом $\varepsilon_{ij}$.
	
\begin{equation}
	|x_{ij}| \le \varepsilon_{ij}
\label{equation:eq1}
% equ(1)
\end{equation}	

Если система $S$ --- блокчейн, то её элементы $S_l$ есть: (а) узлы; (б) кошельки; (в) genesis-блок; (г) остальные блоки.

(а) отдельный узел, соответствующий элементу системы с номером $i$ описывается следующими переменными:
\begin{itemize}
  \item $x_{i1}$ -- отношение откликов узла на обращения системы на текущий момент к числу этих обращений.
  \item $x_{i2}$ -- суммарное вознаграждение $\zeta$, выплаченное этому узлу на текущий момент.
\end{itemize}
Возможны и другие переменные состояния узла.

(б) переменной состояния кошелька $j$ --- номер этого кошелька как элемента системы - $x_{j1}$ является сумма (количество) средств в нём на текущий момент в одной валюте. Если имеется $k$ валют, то следующая переменная $x_{j k+1}$ --- это бинарная переменная, значения которой соответствуют способности кошелька открывать узлы: 0 - не способен, 1 -способен.

(в) Genesis ($G$-блок) имеет смысл назначить элементом системы номер $1$. Тогда переменными состояния $G$-блока будут номера (адреса) корневых узлов, зарегистрированных в этом блоке: $x_{1i_1}, x_{1i_2}, \dots, x_{1i_{\nu}} \:$. Также описываются другие корневые сущности. В совокупности их количество будет равно $k_{l_G} = \dim \mathbb{X}_{l_G} \:$.

(г) Остальные блоки включаются в ``цепочку'' и характеризуются одной переменной $x_{mk_s}$ , значение которой равно в текущий момент времени количеству закрытых блоков в цепочке.

\section{Виды усечения блокчейна}\addcontentsline{toc}{section}{Виды усечения блокчейна}
``Усечение'' системы может быть $\rm{I}$ и $\rm{II}$ рода.
\\
$\rm{I}$. ``Усечение'' $\rm{I}$ рода есть ``усечение'' по времени. Исходя из анализа функционирования системы $S$, выбирается момент времени $t_0 < t^* < t$, такой, для которого координаты системы постоянны на интервале $[t^*, t^* + \delta], \: \delta \ll t_2 - t_1$. Точка $A^* : (x^*_{11},\dots,x^*_{kk_l})$ в момент времени $t^*$ объявляется новым начальным состоянием системы $S$. Для блокчейна в этот момент $t^*$ все блоки закрыты, транзакции не производятся. В $G$-блоке все настройки остаются прежними (вообще говоря, считается, что до усечения все координаты $G$-блока остаются постоянными), т.е. движение системы на $[t_1,t^*]$ есть движение в некоторой гиперплоскости (линейном многообразии размерности $(n-k_{l_G})$, где $l_G$ - номер подпространства $G$-блока). Вся предыстория системы $S$ на интервале $[t_1,t^*)$ отбрасывается, но запоминается.
\\
``Усечение'' $\rm{I}$ рода возможно только тогда, когда система $S$ является \textit{вполне управляемой}. Это значит, что для $\forall$ двух точек пространства $\hat{A}, \tilde{A}\in \mathbb{X}$, существует такое управление (воздействие на блокчейн со стороны создателей), что $S$ может быть переведена из точки $\hat{A}$ в точку $\tilde{A}$. Переход из $\hat{A}$ в $\tilde{A}$ считается успешным, если полностью восстановлены все координаты $S$ в момент $\tilde{t}$, соответствующий $\tilde{A}$.
\\
$\rm{II}$. ``Усечение'' $\rm{II}$ рода есть ``усечение'' по фазовым координатам. Пусть в момент $t^*$ $q$ координат $x_{ij}$ являются ``$\varepsilon_{ij}$-нулевым'' для $\mu$ нулевых блоков с номерами $l_1,\dots,l_{\mu}$. Если принимается решение об ``усечении'' $\rm{II}$ рода, то эти $q$ координат отбрасываются (с запоминанием) и движение системы $S$ продолжается в пространстве $\mathbb{X}_{n-q} \subset \mathbb{X}$ ($\mathbb{X}_{n-q} \rhd \mathbb{X}$). При этом в момент $t^*$ ``скачком'' меняются значения координат $G$-блока, но изменения размерности $k_{l_G}$ подпространства $\mathbb{X}_{l_G}$ не происходит. Координаты $G$-блока меняются на числа равные номерам (адресам) сущностей, оставшихся в $G$-блоке после усечения. Размерность $k_{l_G}$ меняется только в случае изменения количества корневых узлов.
\\
Дальнейшее движение $S$ (при $t>t^*$) может сопровождаться возрастанием размерности пространства состояний до $n$ и более.

\section{Пример усечения блокчейна}\addcontentsline{toc}{section}{Пример усечения блокчейна}
Пусть система $S$ (блокчейн) состоит только из двух элементов 
$l = 1, 2 \; (r = 2) \: $.
$S_1$ --- $G$ - блок. 
$x_{11}$ --- переменная, соответствующая адресу корневого узла, зарегистрированного в этом блоке. 
$G$ - блоку соответствует подпространство $\mathbb{X}_1$, $\dim \mathbb{X}_1 = 1$.
$S_2$ --- кошелёк, где $x_{21}$ --- количество средств в этом кошельке в одной валюте, $x_{22}$ --- бинарная переменная, характеризующая способность кошелька создавать узлы: $x_{22} = 1$ --- кошелёк способен открывать узлы; $x_{22} = 0$ --- кошелёк не способен открывать узлы. Кошельку соответствует подпространство $\mathbb{X}_2$, $\dim \mathbb{X}_2 = 2$.

\begin{equation} 
\mathbb{X} = \mathbb{X}_1 \times \mathbb{X}_2
\label{equation:space_x}
%equ(2)
\end{equation}

Переменные $x_{11}$, $x_{22}$ принадлежат конечным или счётным множествам неотрицательных чисел. 
Переменную $x_{21}$ будем считать принадлежащей множеству неотрицательных чисел. 

Пусть заданы $\varepsilon_{11}$, $\varepsilon_{22}$ -- неотрицательные числа такие, что при выполнении неравенств:

\begin{equation} 
	|x_{11}| \leq \varepsilon_{11}  \; ; \; \; 
	(x_{11}  \leq \varepsilon_{11}) \; ; \; \; 
	|x_{22}| \leq \varepsilon_{22}  \; ; \; \; 
	(x_{22}  \leq \varepsilon_{22})
\label{equation:x_vs_epsilon}
%(3)
\end{equation}
	
Для $x_{21}$ выбрана $\varepsilon_{21} \; $, (в рассматриваемом случае $\varepsilon_{21} = 0 \; $). Если считать что $x_{11} = \mathit{const} = C_1$ ; $x_{22} = \mathit{const} = C_2$ на интервале $t \in [\:t_0, \:\hat{t}\:]$, то траектория движения системы $S$ в пространстве состояний $\mathbb{X}$ будет иметь вид изображённый на рис.~\ref{Traektoriia_S2} кривой 1.

Допустим, что в момент  $\tilde{t}$ изменилось значение переменной $x_{11}$ с $C_1$ на $C_3$ и далее не менялось на $[\:\hat{t}, \: \tilde{t}\:]$, $C_1 < C_3$. Тогда, если на $[\:\hat{t}, \: \tilde{t}\:]$ $x_{21}$ росла, то продолжение траектории $S$ видно на рис.~\ref{Traektoriia_S2} (кривая 2).

\begin{figure}[!h]
\centering
\def\svgwidth{15cm} % изменить размер
\input{fig/f2.pdf_tex}
\caption{Траектория системы из 2 блоков в трёхмерном пространстве состояний.}
\label{Traektoriia_S2}
%fig(2)
\end{figure}

\begin{figure}[!h]
\centering
\def\svgwidth{15cm} % изменить размер
\input{fig/f3.pdf_tex}
\caption{График  координаты $x_{21}$ как функции времени.}
\label{Traektoriia_S3}
%fig(2)
\end{figure}

Пусть в момент $\tilde{t}$ принято решение о ``усечении'' $\rm{II}$ рода блокчейна $S$. Тогда новым начальным состоянием блокчейна объявляется точка $A_2$. Сумма в кошельке есть $x_{21}(\tilde{t})$, способность открывать узлы сохраняется $x_{22}=1$, $G$-блок ``помнит'' только адрес корневого узла $C_3$. Адрес ``старого'' корневого узла $C_1$ и он остаётся доступным в архиве. Движение продолжалось при $t \in [\tilde{t}, \bar{t}]$. Пусть на полуинтервале $[\bar{t},t^*)$ переменные $x_{11}$ и $x_{22}$ не менялись, а $x_{21}$ убывала до $\varepsilon_{21}$ и при $t \ge t^*, x_{21} < \varepsilon_{21}$. В момент $t^*$ $x_{21}$ становится $\varepsilon_{21}$-нулевой и полагается равной нулю (кривая 3). Кошелёк теряет способность открывать узлы и ликвидируется, $x_{22}=0$. Сумма $x_{21}(t^*)$ переводится, например, в системный кошелёк. Производится ``усечение'' блокчейна $\rm{II}$ рода, после которого рассматривается только $\mathbb{X}_1, \dim \mathbb{X}_1=1$ и до изменения $x_{11}$  блокчейн с момента $t^*$ находится в точке с координатами $(C_3,0,0) \in \mathbb{X}$.

\section{Выводы}\addcontentsline{toc}{section}{Выводы}
Представление блокчейна как динамической системы в пространстве состояний позволило рассмотреть вопрос усечения, согласовав свойства системных сущностей, наделённых существенно различными свойствами. Сохранение структуры пространства состояний после усечения его размерности позволяет утверждать, что предложенные подходы к усечению блокчейна дают корректный результат. Усечение блокчейна по времени и по фазовым координатам позволяет существенно экономить ресурсы системы. 
% При данном подходе не учитывается деление узлов на ``толстые'' и ``тонкие'', так как предложенный метод применим к любым сущностям системы, учтённым в пространстве состояний. 
										
\begin{thebibliography} {9}
  \bibitem{nakamoto09}{Satoshi Nakamoto (2009). ``Bitcoin: A Peer-to-Peer Electronic Cash System''. www.bitcoin.org .}
  \bibitem{sumus180427}{a@sumus.team, k@sumus.team, rr@sumus.team. ``Consensus Algorithm for Bigger Blockchain Networks'' (April 27, 2018).}
  \bibitem{zadeh08}{L.A.Zadeh, C.A.Desoer. Linear System Theory: The State Space Approach. Dover Publications, 2008, 656 p.}
\end{thebibliography}

\end{document}
