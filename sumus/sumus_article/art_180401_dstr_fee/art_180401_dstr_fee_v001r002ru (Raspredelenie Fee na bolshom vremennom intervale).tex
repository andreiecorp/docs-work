\documentclass[a4paper, 12pt]{article}
\usepackage[utf8]{inputenc}
\usepackage[T2A]{fontenc}
\usepackage[english, russian]{babel} 
\usepackage{amscd, amsmath, amssymb, amsthm, enumerate, indentfirst, longtable, ifthen} %% addlibrary rom mathematic 
\usepackage{ragged2e}
\usepackage{microtype}
\usepackage{indentfirst}
\usepackage{setspace}
\setstretch{1.15} % interval 

 %%picture 
\RequirePackage{caption}
\DeclareCaptionLabelSeparator{defffis}{. }
\captionsetup{justification=centering,labelsep=defffis}
\usepackage{graphicx}
\graphicspath{{pict/}}
\DeclareGraphicsExtensions{.png,}
 %%end picture 
\usepackage{geometry}
\geometry{
 	tmargin=15mm,         % Up 
 	bmargin=15mm,         % Down 
 	lmargin=15mm,         % left поле
 	rmargin=15mm,         % right  
}

%% new command 
\newcommand{\jj}{\righthyphenmin=20 \justifying}
%opening
\date{Апрель 1, 2018}
\title{Распределение Fee на большом временном интервале }
%%\author{a@sumus.team, k@sumus.team, rr@sumus.team}

\begin{document}

\maketitle

\begin{abstract} 
	Пусть $fee$ - непрерывная случайная величина $\geq 0$.  
\end{abstract}

\section{Введение}\addcontentsline{toc}{section}{Введение}
При функционировании сети блокчейн стремление времени $t$ к $+\infty$ эквивалентно количеству опытов $m$ по достижению консенсуса на текущий момент  $t$. Тогда $t \rightarrow \infty \backsim  m \in \mathbb{N}$. Если в момент $t_m$  определяется мастер-узел и эскорт $m$ раз с начала функционирования сети, то $m$  и есть условное "время". \\

\underline{Задача.} 

Доказать, что при $t \rightarrow +\infty (m \rightarrow +\infty )$ конечный узел из множества $B_n$ получит $fee$  в соответствии с распределением $f(fee)$ и своим "присутствием" в сети. \\

\underline{Решение.}

Будем считать $fee$ непрерывной случайной величиной, распределенной во времени, т.е. случайным процессом. Если в общем случае $fee(t)$ - нестационарный случайный процесс с функцией плотности $f(fee,t)$. В момент времени $t^* f=f(fee,t^*)$ - плотность распределения случайной величины $fee$ с параметрами, значения которы достигнуты к моменту $t^*$ (плотности вероятности в "сечении" $t^*$ случайного процесса $fee(t)$).

Поскольку в силу равномерности распределения номеров узлов в $B_n$ каждый узел в $\forall$ момент $t_m$ может стать мастер-узлом с одинаковой вероятностью $\frac{1}{n}$, $n$ - фиксированное натурально число, то $fee$, получаемая $m$-узлом после закрытия блока зависит только от $f(fee,t^*),t^*=t_n$. Если предположить, что все узлы из $B_n$ \underline{постоянно} находятся в сети, то все они находятся в равном положении и с одинаковой вероятностью получают $fee$ в одном и том же диапазоне (разумеется $m$ должно быть "достаточно велико" $m \gtrsim n$ ) независимо от стационарности или нестационарности случайного процесса $fee(t)$.


Рассмотрим случай стационарного нормального случайного процесса $fee(t)$ с постоянным математическим ожиданием $M_{fee}$ и среднеквадратическим отклонением $\sigma_{fee}$. Тогда $\forall$ узел в момент закрытия блока, соответствующего моменту $t_m$ выбора $m$-узла и эскорта, получает $fee$ в диапазоне $(M_{fee}-3\sigma_{fee} , M_{fee}+3\sigma_{fee})$ с вероятностью $\frac{\Phi(\frac{3}{n})}{n}\approx\frac{0,997}{n}$. Обозначим полученную $fee$ как $fee_m$

Если некоторый узел выбирался $m$-узлом $\overline{m}$ раз к моменту $t_m$, но принял задачу по закрытию блока $\widetilde{m}$ раз $(\widetilde{m} \le  \overline{m})$ то за закрытие текущего блока он получает $\gamma =\frac{\widetilde{m}}{\overline{m}}fee_m$. Остаток $\frac{\overline{m}-\widetilde{m}}{\overline{m}}fee_m$ суммируется со значением $fee_{m+1}$. Это можно назвать - правило мотивации узла. 

В случае нестационарного случайного процесса $fee(t)$ ситуация с постоянно подключёнными узлами в смысле равновероятности получения $fee$ (по сравнению с остальными узлами) не изменится. Если же некоторый узел на каких-то интервалах времени не принимал задание, т.е. был вне сети $(\widetilde{m} <  \overline{m})$, то, например, в случае роста $M_{fee}(t)$ и/или сокращения $\sigma_{fee}(t)$ узел отключавшийся при больших временах чаще, чем при малых, то он потеряет в вознаграждении гораздо больше, чем постоянно подключённые узлы по сравнению со случаем стационарного случайного процесса $fee(t)$. Но это произойдёт только по "вине"  самого узла.



\begin{thebibliography}{1}
	\bibitem{key-1} Satoshi Nakamoto (2009). ``Bitcoin: A Peer-to-Peer
	Electronic Cash System''. www.bitcoin.org .
	
	\bibitem{key-2} (2018.01.10). ``Transactions Speeds: How Do Cryptocurrencies
	Stack Up To Visa or PayPal?''. https://howmuch.net/articles/crypto-transaction-speeds-compared
	. 
	
	\bibitem{key-3} BitFury Group (2015.09.13). ``Proof of Stake versus
	Proof of Work''. http://bitfury.com/content/5-white-papers-research/pos-vs-pow-1.0.2.pdf
	.
	
	\bibitem{key-4} Andrew Miller, Yu Xia, Kyle Croman, Elaine Shi, Dawn
	Song (2016). ``The Honey Badger of BFT Protocols''. https://eprint.iacr.org/2016/199.pdf
	.
	
	\bibitem{key-5} Leslie Lamport, Robert Shostak, Marshall Pease (1982).
	``The Byzantine Generals Problem''. ACM Transactions on Programming
	Languages and Systems. T.4, 3: 382\textendash 401
. 
	
	\bibitem{key-6} National Institute of Standards and Technology. SHA-3
	Standard: Permutation-Based Hash and Extendable-Output Functions.
	https://nvlpubs.nist.gov/nistpubs/FIPS/NIST.FIPS.202.pdf
	.
		
	\bibitem{key-7} S.Josefsson, I.Liusvaara. ``Edwards-Curve Digital
	Signature Algorithm (EdDSA)''. IETF RFC. https://tools.ietf.org/html/rfc8032
	. 
\end{thebibliography}
\end{document}
