%%This is a very basic article template.
%%There is just one section and two subsections.
\documentclass{article}

\usepackage[english, russian]{babel} 
\usepackage[margin=2cm]{geometry}

\title{Десятичная арифметика в двоичных вычислениях с фиксированной точкой.}
\author{engi@sumus.team}
\date{24 июня 2018 г.}

\begin{document}

\maketitle

\begin{abstract}
В некоторых областях применения вычислительной техники традиционно используется десятичная система счисления. 
Сама вычислительная техника, в подавляющем большинстве аппаратуры и моделей программирования, использует системы счисления, базирующиеся на двоичной системе счисления. 
Рассмотрим некоторые проблемы и их решения, связанные с переходом представления чисел в десятичной системе счисления к вычислениям в двоичной системе счисления. 
Будут рассмотрены 4 арифметических действия: сложение, вычитание, умножение и деление с округлением. 
\end{abstract}

\section{Запись и внутреннее представление числа.}
Запись числа, как десятичного, является ещё и набором методов алгоритмов и правил для выполнения арифметических действий.
Например, правило сложения многоразрядных чисел, ``деление в столбик'' и другие алгоритмы для десятичных чисел.
Вычислительные машины используют двоичную систему счисления и отличающиеся от десятичных правила и алгоритмы.

\section{Вычисления с целыми числами.}
Запись целого неотрицательного числа в любой позиционной системе счисления:

\begin{equation}
x = \sum\limits_{k=0}^{n-1} a_k b^k
\end{equation}

При выполнении операций ($+$, $-$, $*$) не возникает проблем перевода чисел в любую другую систему счисления. 
Деление с округлением подразумевает получение дробной части, возможно, бесконечной периодической. 
Такая часть для систем счисления с чётным основанием округляется так: если первая цифра дробной части меньше половины основания системы счисления (1 для двоичной, 5 для десятичной) то целая часть результата остаётся без изменений, в другом случае - целая часть результата увеличивается на 1.
\\
Запись отрицательных чисел выглядит как префиксирование числа знаком ``$-$''. Внутренним представлением целого отрицательного числа в большинстве моделей программирования будет ``дополнительное''. 
При таком представлении, сложение и вычитание чисел со смешанными знаками выглядит как те же действия с неотрицательными числами. Переполнение результата выглядит как неожидаемый знак результата.
Умножение и деление чисел со смешанными знаками в ``дополнительном'' представлении требует приведения их неотрицательный вид, а затем восстановление знака в результате.

\section{Вычисления с числами с фиксированной точкой.}
Есть несколько подходов для работы с числами десятичной записи:
\begin{itemize}
\item Двоично-десятичное представление.
\item Двоичное представление.
\item Двоичное представление с контролем дробной части.
\end{itemize}

\subsection{Двоично-десятичное представление.}
Такое представление максимально приближено к целевому. 
Соответственно, при таком представлении не возникает проблем перевода между системами счисления для записи. 
Усложнением при работе с таким представлением будет отсутствие двоично-десятичной арифметики в современных моделях программирования. 
Выполнение действий нужно будет производить индивидуально по каждой паре цифр, что существенно увеличивает вычислительные затраты. 
Кстати, большинство ``бухгалтерских'' калькуляторов работает именно в таком представлении чисел.

\subsection{Двоичное представление.}
Это представление максимально приближено к ``машинному''. 
Все внутренние операции выполняются минимальных количеством машинных действий.
При попытке совместить двоичное представление и десятичную запись чисел с дробной частью возникают проблемы. 
Например, 0.1 десятичной записи, в двоичной записи с фиксированной точкой превратятся в бесконечную периодическую дробь 0.0(0011), что безусловно приводит к потере точности.
Подобную ситуацию можно ``залатать'' алгоритмом с избыточной точностью представления/вычисления и предварительным округлением при записи числа в десятичном 
\footnote{Есть подозрение, что в Excel так и сделано.}. 
Однако, такой метод не решает проблему полностью. 
При определённых вычислениях, предварительно округлённые в одну сторону числа могут дать в результате число, отличающееся на 1 в младшем разряде.

\subsection{Двоичное представление с контролем дробной части.}
Предлагается представление числа, аналогично двоичному, но с набором коррекций по всем арифметическим действиям. 
Дополнительным условием упрощения набора операций будет использование дробных чисел единого формата, с десятичной точкой перед фиксированным количеством цифр.
Для числа с фиксированной десятичной точкой при известном количестве цифр в дробной части введём скрытый делитель:
\[
\begin{array}{lr}
	x = X / D \\
	D = 10^d	
\end{array}
\]
где:\\
$x$ –- число с фиксированной десятичной точкой.\\
$X$ –- целое число.\\
$D$ –- делитель.\\
$d$ -- количество цифр после десятичной точки.

\subsubsection{Приведение десятичной записи.}
Для перевода числа из десятичной записи:
\begin{enumerate}
\item Добавить нули справа до получения количества цифр  после десятичной точки.
\item Перевести число из десятичной записи как целое стандартным преобразованием.
\end{enumerate}

Для перевода числа в десятичную запись:
\begin{enumerate}
\item Перевести число в десятичную запись как целое стандартным преобразованием.
\item ``Поставить'' десятичную точку в позицию  справа и отбросить нули справа.
\end{enumerate}

\subsubsection{Приведение целого числа к дробному виду.}
Чтобы привести целое число к дробному виду нужно умножить его на делитель .

\subsubsection{Округление дробного числа.}
Для округления дробного числа до указанного количества цифр  после десятичной точки нужно:
	
\begin{enumerate}
\item Разделить  на  с округлением результата.
\item Умножить  на .
\end{enumerate}

\subsubsection{Сложение/вычитание.}
Сложения/вычитания двух дробных чисел производится также как и целых чисел.
Сложения/вычитания дробного и целого чисел требует приведения целого числа к дробному виду.

\subsubsection{Умножение.}
Формула:
\\
Показывает необходимость приведения результата. Умножение выполняется так:
\begin{enumerate}
\item Перемножить представления чисел с получением результата удвоенной разрядности.
\item Произвести целочисленное деление с округлением предыдущего результата на .
\item Отбросить ``лишние'' старшие разряды для получения требуемой разрядности представления дробного числа.
\end{enumerate}

\subsubsection{Деление.}
Формула:
\\
Показывает ход выполнения действий. Деление выполняется так:
\begin{enumerate}
  \item Помножить делимое на  с получением результата увеличенной разрядности.
  \item Произвести целочисленное деление с округлением – предыдущего результата на делитель.
  \item Отбросить ``лишние'' старшие разряды для получения требуемой разрядности представления дробного числа.
\end{enumerate}

\section{Реализация.}
Современная реализация модели ``двоичное представление с контролем дробной части'' сделана в Java-классе java.math.BigDecimal . Удобными возможностями этого класса являются индивидуальное задание десятичной точки (scale в терминологии библиотеки) и широкий диапазон представляемых чисел.
Представляется несложным записать набор классов/функций выполняющих требуемые операции для чисел с предопределенным форматом.

\section{Выполнение операций с длинными целыми.}

\subsection{Представление.}
Представление длинного целого – это массив целых примитивов, например, массив long.

\subsection{Знак числа и смена знака.}
Признаком отрицательного числа будет ``1'' в старшем бите старшего слова.
Ноль не отличается по знаку от положительных чисел.
Для смены знака:
\begin{itemize}
  \item Бит-инверсия.
  \item Инкрементирование.
\end{itemize}
ВНИМАНИЕ! При смене знака максимального по модулю отрицательного числа происходит переполнение.

\subsection{Сложение и вычитание.}

\subsubsection{Сложение.}
Сложение по словам от младшего к старшему, с учётом переноса. Перенос при сложении двух целых одной разрядности и бита переноса из предыдущей суммы можно вычислить так:
\begin{itemize}
  \item Если оба старших бита ``1'', то перенос будет.
  \item Если оба старших бита ``0'', то переноса не будет.
  \item Если только один старший бит ``1'', то перенос будет если старший бит суммы ``0'' и не будет переноса, если старший бит суммы ``1''.
  \item Перенос ``снизу'' при сложении младших слов равен ``0''.
  \item Перенос ``вверх'' от сложения старших слов игнорируется.
\end{itemize}

\subsubsection{Вычитание.}
Для вычитания можно использовать два подхода:
\begin{itemize}
  \item Использовать сложение со сменой знака вычитаемого.
  \item Произвести вычитание с введением дополнительной базовой функции.
\end{itemize}
Предлагаю использовать минимум базовых функций, соответственно ``прямое'' вычисление разности не приводится.

\subsubsection{Переполнение результата при сложении и вычитании.}
Переполнение результата при сложении двух целых одного знака – это смена знака результата.
Переполнение результата при сложении двух целых разных знаков не возникает.
Так как вычитание производится с помощью сложения, то переполнение результата происходит, так же как и при сложении.

\end{document}
