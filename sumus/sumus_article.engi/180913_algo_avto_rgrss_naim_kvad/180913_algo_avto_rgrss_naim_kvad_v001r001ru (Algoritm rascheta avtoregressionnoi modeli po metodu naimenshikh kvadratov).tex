\documentclass[]{article}
\usepackage[english, russian]{babel} 
\usepackage[margin=2cm]{geometry}
\usepackage{indentfirst} % Красная строка после заголовка

\title{Алгоритм расчёта авторегрессионной модели по методу наименьших квадратов.}
\author{engi@sumus.team}
\date{18 сентября 2018 г.}

\begin{document}

\maketitle

%\begin{abstract}
%\end{abstract}

%%%%%%%%%%%%%%%%%%%%%%%%%%%%%%%%%%%%%%%%%%%%%%%%%%%%%%%%%%%%%%%%%%%%%%%%%%%%%%%%

\section{Алгоритм.}

Согласно определению, авторегрессионный процесс порядка $p$ ($AR(p)$ - процесс) выглядит так:

\begin{equation}
X_t = c + \sum\limits_{i=1}^{p} a_i X_{t-i} + \epsilon_t
\end{equation}

где $a_1, \ldots , a_p$ - параметры модели (коэффициенты авторегрессии), $c$ - постоянная (часто для упрощения предполагается равной нулю), а $\epsilon_t$ - белый шум.
Для использования авторегрессионной модели в качестве прогноза, выберем формулу:

\begin{equation}
Y_n = a_0 + \sum\limits_{i=1}^{p-1} a_i x_{n-i}
\end{equation}

где для получения очередного значения $Y_n$, нам известны значения $x_{n-i}$ не новее  $x_{n-i}$.
Метод наименьших квадратов подразумевает минимизацию функции разности квадратов целевой функции и аппроксимирующей на некотором интервале. В качестве приближения вместо функций используем последовательности. \\

Одиночное значение квадрата разности:

\begin{equation}
\sigma_n = (X_n + Y_n)^2
\end{equation}

Минимизируемая функция:

\begin{equation}
\Delta_n = \sum\limits_{j=0}^{N-1} \sigma_{n - j}
\end{equation}

где $N$ - количество точек для минимизации. $N$ должно быть существенно больше количества коэффициентов $p$. \\
Для вычисления коэффициентов $a_k$ произведем взятие частных производных по всем коэффициентам $a_k$ и приравняем полученные формулы к нулю. Получим систему линейных уравнений вида:

\begin{equation}
	\bigcup_{k=0}^{p-1} = 
   	\left\{
	\begin{array}{lr}
   		k=0, \; 
   			a_0 N + \sum\limits_{i=1}^{p-1} a_i \left( \sum\limits_{j=0}^{N-1} X_{n-i-j} \right) 
   			= \sum\limits_{j=0}^{N-1} X_{n-j} \\
		k>0, \; 
			a_0 \sum\limits_{j=0}^{N-1} X_{n-k-j} + \sum\limits_{i=1}^{p-1} a_i \left( \sum\limits_{j=0}^{N-1} X_{n-i-j} X_{n-k-j} \right) 
			= \sum\limits_{j=0}^{N-1} X_{n-j} X_{n-k-j}
	\end{array}
	\right.
\end{equation}

%%%%%%%%%%%%%%%%%%%%%%%%%%%%%%%%%%%%%%%%%%%%%%%%%%%%%%%%%%%%%%%%%%%%%%%%%%%%%%%%

\section{Анализ реализации.}

\par
Произведен расчет прогноза по данному алгоритму на наборах реальных данных. 
Были использованы ежедневные курсы SDR к трём основным валютам: USD, GBP, CAD. 
Полученные результаты оказались независимы от набора данных.
\par
При задание порядка авторегресии выше 8, параметры модели становятся неустойчивыми - возникают ситуации неразрешимости системы уравнений.
\par
Свободный член - коэффициент $a_0$, а также коэффициенты выше 2 порядка, выглядят незначительными.
Соответственно, функция "прогноза" существенно похожа на аппроксимируемую функцию, задержанную на 1 отсчёт.
При этом, "прогноз" порядка выше 1 оказывается всегда лучше "прогноза" порядка 1 как по максимальному отклонению, так и по интегральному отклонению.

\end{document}
