\documentclass[12pt,twoside,a4paper]{article}

\usepackage[utf8]{inputenc} % Включаем поддержку UTF8
\usepackage[russian]{babel} % [english, russian]

\usepackage[margin=2cm]{geometry}
\usepackage{indentfirst} % Красная строка после заголовка

\usepackage{enumitem} % расширение для enumerate
%\makeatletter
%\AddEnumerateCounter{\asbuk}{\russian@alph}{щ}
%\makeatother

\usepackage{float}

% overset
\usepackage{amsmath}


\usepackage{graphicx,xcolor} % графика для svg
\graphicspath{{fig/}}

\title{Обмен ключами с использованием технологии \textbf{blockchain}.}
\author{engi@sumus.team}
\date{16 июля 2020 г.}

\begin{document}

\maketitle

\section{Постановка задачи}

\subsection{Термины}

\begin{itemize}
	\item \textit{Консьюмер} --- сущность, запрашивающая выполнение вычислительной задачи.
	\item \textit{Продюсер} --- сущность, предоставляющая ресурс для выполнения вычислительной задачи.
	\item \textit{blockchain} --- сетевая база данных, не допускающая изменение уже внесённых записей.
\end{itemize}

\subsection{Процесс}

\begin {enumerate}[label=\alph*)]
	\item \textit{Консьюмер} размещает запись $RC_1$ запрашивающую выполнение вычислительной задачи. 
	\item \textit{Продюсер} получает запись $RC_1$ и размещает запись $RP_{A1}$ о намерении предоставить ресурс для выполнения запрошенной вычислительной задачи. 
	\item \textit{Консьюмер} получает запись $RP_{A1}$ и размещает запись $RC_{A1}$, подтверждающую начало выполнения вычислительной задачи. 
\end{enumerate}

\subsection{Задача}

Требуется дополнить содержимое записей $RC_1$, $RP_{A1}$, $RC_{A1}$ такими полями, чтобы выполнялись условия:

\begin{itemize}
	\item Количество шагов обмена должно быть минимальным, желательно не более 3 -- по количеству шагов \textit{процесса}.
	\item В результате \textit{процесса}, \textit{консьюмер} и \textit{продюсер} должны получить одинаковое число, недоступное другим обозревателям \textit{blockchain}.
	\item Весь обмен при выполнении \textit{задачи} должен происходить через \textit{blockchain}.
\end{itemize}

\section{Предлагаемое решение №1}

\subsection{Положения}

Существует обратимое криптографическое преобразование с ключом, обладающее свойством коммутативности \footnote{например, алгоритм RSA}. Обозначим пару шифрование-расшифрование таких преобразований:

\begin{equation}
	\begin{split}
	n \xrightarrow{ \mathcal{F}      (k_a) } n_a \\
	n_a \xrightarrow{ \mathcal{F} ^{-1}(k_a) } n 
	\end{split}
	\label{equ_1}
\end{equation}

Коммутативностью называется свойство преобразования для которого, при многократном шифровании, для расшифрования не важен порядок обратных преобразований.

\begin{equation}
	\begin{split}
	n \xrightarrow{\mathcal{F}(k_a)} n_a \xrightarrow{\mathcal{F}(k_b)} n_{ab} \\
	n_{ab} \xrightarrow{\mathcal{F}^{-1}(k_b)} n_a \xrightarrow{\mathcal{F}^{-1}(k_a)} n \\
	n_{ab} \xrightarrow{\mathcal{F}^{-1}(k_a)} n_b \xrightarrow{\mathcal{F}^{-1}(k_b)} n \\
	\end{split}
	\label{equ_1}
\end{equation}

\subsection{Алгоритм}

\begin {enumerate}[label=\alph*)]
	\item \textit{Консьюмер} создаёт: \\ 
	ключ $k_a$, \\
	число $n$
	\item \textit{Консьюмер} шифрует число $n$ \\
	$n \xrightarrow{\mathcal{F}(k_a)} n_a$, \\
	заносит $n_a$ в $RC_1$.
	\item \textit{Продюсер} создаёт: \\ 
	ключ $k_b$
	\item \textit{Продюсер} получает число $n_a$ из записи $RC_1$, шифрует \\ 
	$n_a \xrightarrow{\mathcal{F}(k_b)} n_{ab}$, \\
	заносит $n_{ab}$ в $RP_{A1}$.
	\item \textit{Консьюмер} получает число $n_{ab}$ из записи $RP_{A1}$, расшифрует \\ 
	$n_{ab} \xrightarrow{\mathcal{F}^{-1}(k_a)} n_b$, \\
	заносит $n_b$ в $RC_{A1}$.
	\item \textit{Продюсер} получает число $n_b$ из записи $RC_{A1}$, расшифрует \\ 
	$n_b \xrightarrow{\mathcal{F}^{-1}(k_b)} n$
\end{enumerate}

В результате обмена \textit{консьюмер} и \textit{продюсер} получают число $n$ недоступное другим обозревателям \textit{blockchain}.

\end{document}
